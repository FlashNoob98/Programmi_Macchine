\documentclass[a4paper,12pt]{article}
\usepackage[T1]{fontenc}
\usepackage[utf8]{inputenc}
\usepackage{mathtools}
\usepackage[italian]{babel}
%\usepackage{graphicx}
\usepackage{float}
\usepackage{textcomp}
\usepackage{amsmath}

\title{Esercitazione 6}
\author{Olivieri Daniele}
\date{}

\begin{document}
\maketitle
Valutare la portata a regime in un circuito idraulico nel quale una pompa sposta un liquido tra due serbatoi posti con una differenzza di quota
di 12 metri e nei quali ci siano le pressioni di 1 bar in quello inferiore e 1.6 bar in quello superiore utilizzando
\begin{enumerate}
    \item Una pompa centrifuga di diametro 129 mm, con flangia di aspirazione di 65 mm e una velocità di 2900 rpm.
    \item Una pompa a palette (volumetrica)
\end{enumerate}

\section{Pompa centrifuga}
È necessario in primo luogo dimensionare il circuito idraulico, i dati a disposizione sono:
\begin{itemize}
    \item il diametro della flangia di aspirazione
    \item la differenza di quota da raggiungere
\end{itemize}
% La pressione idrostatica che devono sopportare le tubazioni si può calcolare con la legge di Stevino
% \begin{equation}
%     P = \rho g h
% \end{equation}
% che per l'acqua ad una quota di 12 metri vale 1.17 bar alla quale va aggiunta la differenza di pressione 
Con un diametro nominale della flangia di 65 mm possiamo utilizzare un tubo in acciaio con diametro nominale di 2" 1/2 (2 pollici e mezzo) e della lunghezza minima di 12 m.
Utilizzando il catalogo del produttore \textit{Oppo} possiamo trovare tabellati i valori di velocità e perdita di carico per unità di lunghezza.
La \textbf{prevalenza} del circuito si può calcolare con la seguente:
\begin{equation}
    H_{Tc} = \frac{P_2-P_1}{\rho g} + (z_2-z_1) + \frac{c_2^2-c_1^2}{g2} + \frac{\Delta_{P_{12}}}{\rho g}
\end{equation}
Il termine contenente la differenza di velocità può essere trascurato data la presenza dei serbatoi.
Il valore della prevalenza è quindi 18.13 m al quale va aggiunta la perdita di carico, che dipende dalla portata del fluido.
Da metri cubi orari a litri al secondo bisogna moltiplicare per 3.6.

Seguendo la curva della pompa è possibile confrontare il valore di prevalenza totale rispetto a quello minimo appena calcolato.
Con una portata di 15 $m^3/s$ o 54 $l/h$ si ha una prevalenza totale di 23.5 metri, facendo la differenza con quella iniziale sappiamo che
il condotto dovrà avere una lunghezza tale da dissipare una prevalenza aggiuntiva di 5.37 metri.

Sfruttando le tabelle del produttore vediamo che con una portata di 15 m/s il tubo in acciaio ha una perdita di 265.51 m/km, dividendo la prevalenza aggiuntiva
per questo valore si ottiene la lunghezza del circuito pari a 20 metri, superiore ai 12 metri di dislivello tra i due serbatoi.

\section{Pompa volumetrica}
La pompa volumetrica funziona con portate molto più basse; fissata la lunghezza del circuito precedente, infatti, le perdite legate alla portata nelle tubazioni sono limitate.
% per una portata di 1 l/s si ha infatti una prevalenza del circuito pari a 18.165 metri mentre per una portata di 1.5 l/s la prevalenza vale 18.205 metri.
% Questi valori sono stati ricavati dalla stessa tabella precedente presente sul sito del produttore e devono essere convertiti nel sistema internazionale per poter essere
% utilizzati nel grafico di funzionamento della pompa.
L'energia posseduta da un fluido posto ad una certa altezza è pari a 
\begin{equation}
    U = mgh
\end{equation}
Data la prevalenza espressa in metri è possibile ricavare l'energia specifica moltiplicando la prevalenza per l'accelerazione gravitazionale $g$ e dividendo per 1000 si ricava 
l'energia espressa in kJ/kg, come in tabella. La portata espressa in l/s va moltiplicata per 60 per ottenere i l/min.
Si ricavano i seguenti valori
\begin{center}
    \begin{tabular}{c|c|c|c}
        Q (lt/min)  &   Q (lt/s)&H(m)   &H (kJ/kg)  \\ \hline
        60          &   1       &18.165 &0.1782     \\ \hline
        90          &   1.5     &18.205 &0.1785     \\ \hline
        120         &   2       &18.258 &0.1791     \\ \hline
    \end{tabular}
\end{center}
\end{document}
