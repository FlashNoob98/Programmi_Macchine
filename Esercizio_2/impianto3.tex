\section{Impianto da 18 MW con 4 spillamenti}
L'ultimo impianto presenta 4 spillamenti per il preriscaldamento del fluido, la potenza netta è necessariamente diminuita
ma un rapido calcolo permette di valutare il rendimento elettrico globale aumentato ad un valore di 30.01\%
Il rendimento dell'alternatore si può stimare, anche in questo caso, pari a 97.5\%.
Il rendimento globale vale dunque 30.78\%

\subsection{Rendimento di combustione}
Potenza termica ricavabile dal combustibile: $\dot m_c \cdot H_i = 59.850$ MW.\\
Potenza termica trasferita al fluido: $\dot m_{3'} \cdot (h_{3'} - h_{2R}) = 53.916$ MW.\\
Rendimento di combustione: 90.08\%.

\subsection{Rendimento limite}
seguiamo ora l'analisi del ciclo limite considerando come valori di riferimento quelli dello stato 1 e 3.
\begin{center}
    \begin{tabular}{r|c|c|c}
        stato    & $p\ (atm)$ & $h\ (kJ/kg)$ & $T\ (\text{°}C) $\\ \hline
        1   &        0.0419 &          125.6   &           30     \\ \hline
        3   &        59.05  &           3381   &           485
    \end{tabular}
\end{center}
Siano SP1, SP2, SP3 ed SP4 gli stati termodinamici consecutivi ai vari spillamenti, si suppone che abbiano tutti la stessa entropia specifica.


