\section{Impianto da 18 MW con 4 spillamenti}
L'ultimo impianto presenta 4 spillamenti per il preriscaldamento del fluido, la potenza netta è necessariamente diminuita
ma un rapido calcolo permette di valutare il rendimento elettrico globale aumentato ad un valore di 30.01\%
Il rendimento dell'alternatore si può stimare, anche in questo caso, pari a 97.5\%.
Il rendimento globale vale dunque 30.78\%

\subsection{Rendimento di combustione}
Potenza termica ricavabile dal combustibile: $\dot m_c \cdot H_i = 59.850$ MW.\\
Potenza termica trasferita al fluido: $\dot m_{3'} \cdot (h_{3'} - h_{2R}) = 53.916$ MW.\\
Rendimento di combustione: 90.08\%.

\subsection{Rendimento limite}
seguiamo ora l'analisi del ciclo limite considerando come valori di riferimento quelli dello stato 1 e 3.
\begin{center}
    \begin{tabular}{r|c|c|c}
        stato    & $p\ (atm)$ & $h\ (kJ/kg)$ & $T\ (\text{°}C) $\\ \hline
        1   &        0.0419 &          125.6   &           30     \\ \hline
        3   &        59.05  &           3381   &           485
    \end{tabular}
\end{center}
Siano SP1, SP2, SP3 ed SP4 gli stati termodinamici consecutivi ai vari spillamenti, si suppone che abbiano tutti la stessa entropia specifica.
Sia fissata ancora una volta la pressione all'interno del mescolatore adiabatico uguale a quella del secondo spillamento, ossia 5.32 atm.
Si può ricavare lo stato 1' consecutivo alla compressione isoentropica provocata dalla pompa di estrazione.

Calcolo quindi gli stati termodinamici degli spillamenti, l'entropia è identica a quella dello stato 3 in ingresso alla turbina.
\begin{center}
    \begin{tabular}{l|c|c|c|c|c}
        stato    & $p\ (atm)$ & $T\ (\text{°}C) $&$X\ (\%$)& $h\ (\frac{kJ}{kg})$  & $s\ (\frac{kJ}{kg\cdot K})$\\ \hline
        SP1   &       8.42 &     211     & vap. surr. &  2863   &6.837 \\ \hline 
        SP1'    &       8.42    &173    &   liq. sott. &732.2   &2.071  \\ \hline
        SP2  &        5.32    &    162    & vap. surr. & 2769  &6.837 \\ \hline    
        SP3  &      1.47       &    111    &  93.30    &2544 &6.837   \\ \hline
        SP4   &    0.55      &      84    &    88.83   & 2393&6.837 
    \end{tabular}
\end{center}

Anche lo stato 4 si può ricavare fissando l'entropia e la pressione successive alla espansione in turbina.

Supponiamo che lo stato 2R sia ad una temperatura pari alla temperatura di saturazione dello spillamento 1 (SP1) pari a 173°C.

Ipotizzando uno scambio termico ideale, posso suppore che le temperature dei due fluidi uscenti dal rigeneratore siano identiche, ossia che la temperatura
del fluido spillato dalla turbina, dopo lo scambio termico, abbia la stessa temperatura del fluido in ingresso al generatore di vapore (173°C).

Per valutare la temperatura in uscita dalla pompa (stato 2) posso eseguire un bilancio energetico sul rigeneratore.
\begin{equation*}
    h_2 =\frac{ \dot m_{2R}\cdot h_{2R} - \dot m_{SP2}\cdot(h_{SP2}-h_{SP2'})}{\dot m_{2R}} = 700.6\ kJ/kg
\end{equation*}
Ricavato lo stato 2, si può ricavare lo stato 2' considerando la compressione isoentropica.

\begin{center}
    \begin{tabular}{l|c|c|c|c|c}
        stato    & $p\ (atm)$ & $T\ (\text{°}C) $&$X\ (\%$)& $h\ (\frac{kJ}{kg})$  & $s\ (\frac{kJ}{kg\cdot K})$\\ \hline
        1   &        0.0419 &            30   &    0      & 125.6    &0.4368 \\ \hline     %uscita condensatore 
        1'  &        5.32    &           30.01   &liq. sott.  & 126.3&0.4368 \\ \hline      %uscita pompa estrazione
        2'  &          5.32  &         154&liq.sott.            &694&1.9866  \\ \hline      %ingresso pompa alimento
        2   &          59.05&           165&liq. sott.      &700.6& 1.9866   \\ \hline      %uscita pompa alimento
        2R  &       59.05   &           173     &liq. sott. &  735  & 2.0644 \\ \hline      %ingresso caldaia  
        3   &        59.05  &            485&  vap. surr.    &  3381 &6.837  \\ \hline      %ingresso turbina
        4   &        0.0419       &      29.76&     79.81 &    2064  &6.837                %uscita turbina
    \end{tabular}
\end{center}
