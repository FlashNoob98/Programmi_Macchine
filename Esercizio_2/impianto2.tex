\section{Secondo impianto da 19.69 MW}
Il secondo impianto prevede rispetto al primo uno \textbf{spillamento} dalla turbina, ossia una sottrazione di vapore caldo da uno degli stadi della turbina, 
utile al preriscaldamento del fluido di lavoro prima dell'ingresso nel generatore di vapore.
La potenza elettrica generata è quindi diminuita ma è diminuita anche la portata di combustibile, è necessario quindi rieseguire il calcolo dei rendimenti.
\subsection{Rendimento globale}
Il rendimento globale elettrico calcolabile con la \eqref{eq:rendimento_elettrico} vale 27.66\%.

Considerando il rendimento dell'alternatore identico al caso precedente, il rendimento globale d'impianto è pari a 28.36\%
\subsection{Rendimento di combustione}
La potenza termica trasferita al fluido nel generatore di vapore è valutabile con il suo salto entalpico e pari a 63.10 MW.
Il calore prodotto dal combustibile, calcolato in precedenza per i rendimenti globali è pari a 70.29 MW.
Il rendimento di combustione è quindi pari al 89.77\%.

\subsection{Rendimento limite}
Eseguiamo ora l'analisi del ciclo limite considerando come valori di riferimento quelli dello stato 1 e 3.
\begin{center}
    \begin{tabular}{r|c|c|c}
        stato    & $p\ (atm)$ & $h\ (kJ/kg)$ & $T\ (\text{°}C) $\\ \hline
        1   &        0.0419 &          125.6   &           30     \\ \hline
        3   &        59.05  &           3381   &           485
    \end{tabular}
\end{center}
Considero SP2 lo stato termodinamico successivo allo spillamento dalla turbina, con una portata massica di 0.725 kg/s. Ricalcolo la tabella degli stati.
Non è possibile calcolare immediatamente lo stato 2' dato che è successivo ad un mescolamento adiabatico con il vapore spillato dalla turbina, calcolo quindi prima
lo stato 4 in uscita dalla turbina considerando la trasformazione isoentropica e la pressione identica alla pressione misurata nell'impianto reale.
Anche l'entropia del vapore spillato sarà quindi pari a quella dello stato 3 e 4 ossia 6.837 kJ/kgK, ricaviamo gli altri valori dello stato.

Per permettere un corretto mescolamento, la pompa di estrazione deve portare l'acqua dallo stato 1 ad uno stato 1' con una pressione di 5.32 atm, il processo si assume
isoentropico.

Considerando la trasformazione adiabatica nell'estrattore di vapore è possibile fare un bilancio sull'entalpia:
\begin{equation*}
    h_{2'} = \frac{\dot m_{1'}\cdot h_{1'} + \dot m_{SP2} \cdot h_{SP2}}{\dot m_{2'}} = 222.5\ kJ/kg
\end{equation*}
Considerata la pressione dello stato 2' identica a quella del vapore spillato e calcolata l'entalpia si possono ricavare le altre variabili di stato.

Lo stato 2 è successivo alla pompa di alimento, trasformazione considerata isoentropica, fissiamo il valore di pressione identico a quello dello stato 3 e calcoliamo
le altre variabili.
\begin{center}
    \begin{tabular}{l|c|c|c|c|c}
        stato    & $p\ (atm)$ & $T\ (\text{°}C) $&$X\ (\%$)& $h\ (\frac{kJ}{kg})$  & $s\ (\frac{kJ}{kg\cdot K})$\\ \hline
        1   &        0.0419 &            30   &    0      & 125.6    &0.4368 \\ \hline
        1'  &        5.32    &           30.01   &liq. sott.  & 126.3&0.4368 \\ \hline
        2'   &       5.32    &           53.1&liq. sott.& 222.5    &0.743 \\ \hline
        2   &        59.05   &           53.1&liq. sott.&  228.0   &0.743   \\ \hline
        3   &        59.05   &           485  & vap. surr.& 3381     &6.837  \\ \hline
        SP2  &       5.32        &       160.9 &  vap. surr.&2767     &6.837  \\ \hline
        4   &        0.0419       &      29.76&     79.81 & 2064     &6.837
    \end{tabular}
\end{center}



