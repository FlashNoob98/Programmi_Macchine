\section{Secondo impianto da 19.69 MW}
Il secondo impianto prevede rispetto al primo uno \textbf{spillamento} dalla turbina, ossia una sottrazione di vapore caldo da uno degli stadi della turbina, 
utile al preriscaldamento del fluido di lavoro prima dell'ingresso nel generatore di vapore.
La potenza elettrica generata è quindi diminuita ma è diminuita anche la portata di combustibile, è necessario quindi rieseguire il calcolo dei rendimenti.
\subsection{Rendimento globale}
Il rendimento globale elettrico calcolabile con la \eqref{eq:rendimento_elettrico} vale 27.66\%.

Considerando il rendimento dell'alternatore identico al caso precedente, il rendimento globale d'impianto è pari a 28.36\%
\subsection{Rendimento di combustione}
La potenza termica trasferita al fluido nel generatore di vapore è valutabile con il suo salto entalpico e pari a 63.10 MW.
Il calore prodotto dal combustibile, calcolato in precedenza per i rendimenti globali è pari a 70.29 MW.
Il rendimento di combustione è quindi pari al 89.77\%.

\subsection{Rendimento limite}
\label{subsec:rendimento_limite_2}
Eseguiamo ora l'analisi del ciclo limite considerando come valori di riferimento quelli dello stato 1 e 3.
\begin{center}
    \begin{tabular}{r|c|c|c}
        stato    & $p\ (atm)$ & $h\ (kJ/kg)$ & $T\ (\text{°}C) $\\ \hline
        1   &        0.0419 &          125.6   &           30     \\ \hline
        3   &        59.05  &           3381   &           485
    \end{tabular}
\end{center}
Considero SP2 lo stato termodinamico successivo allo spillamento dalla turbina, con una portata massica di 0.725 kg/s. Ricalcolo la tabella degli stati.

Non è possibile calcolare immediatamente lo stato 2' dato che è successivo ad un mescolamento adiabatico con il vapore spillato dalla turbina, calcolo quindi prima
lo stato 4 in uscita dalla turbina considerando la trasformazione isoentropica e la pressione identica alla pressione misurata nell'impianto reale.
Anche l'entropia del vapore spillato sarà quindi pari a quella dello stato 3 e 4 ossia 6.837 kJ/kgK, ricaviamo gli altri valori dello stato.

Per permettere un corretto mescolamento, la pompa di estrazione deve portare l'acqua dallo stato 1 ad uno stato 1' con una pressione di 5.32 atm, il processo si assume
isoentropico.

Considerando la trasformazione adiabatica nell'estrattore di vapore è possibile fare un bilancio sull'entalpia:
\begin{equation*}
    h_{2'} = \frac{\dot m_{1'}\cdot h_{1'} + \dot m_{SP2} \cdot h_{SP2}}{\dot m_{2'}} = 222.5\ kJ/kg
\end{equation*}
Considerata la pressione dello stato 2' identica a quella del vapore spillato e calcolata l'entalpia si possono ricavare le altre variabili di stato.

Lo stato 2 è successivo alla pompa di alimento, trasformazione considerata isoentropica, fissiamo il valore di pressione identico a quello dello stato 3 e calcoliamo
le altre variabili.
\begin{center}
    \begin{tabular}{l|c|c|c|c|c}
        stato    & $p\ (atm)$ & $T\ (\text{°}C) $&$X\ (\%$)& $h\ (\frac{kJ}{kg})$  & $s\ (\frac{kJ}{kg\cdot K})$\\ \hline
        1   &        0.0419 &            30   &    0      & 125.6    &0.4368 \\ \hline     %uscita condensatore 
        1'  &        5.32    &           30.01   &liq. sott.  & 126.3&0.4368 \\ \hline      %uscita pompa estrazione
        2'   &       5.32    &           53.1&liq. sott.& 222.5    &0.743 \\ \hline         %uscita mescolatore
        2   &        59.05   &           53.1&liq. sott.&  228.0   &0.743   \\ \hline       %uscita pompa alimento
        3   &        59.05   &           485  & vap. surr.& 3381     &6.837  \\ \hline  %ingresso turbina
        SP2  &       5.32        &       160.9 &  vap. surr.&2767     &6.837  \\ \hline  %spillamento
        4   &        0.0419       &      29.76&     79.81 & 2064     &6.837                 %uscita turbina
    \end{tabular}
\end{center}

È ora possibile calcolare il rendimento limite definito nella \eqref{eq:redimento_limite}. Per conoscere la potenza limite eseguo un'analisi energetica
ai flussi di massa entranti e uscenti dalla turbina:
\begin{equation*}
    P_T = \dot m_3 \cdot h_3 - \dot m_1 \cdot h_1 - \dot m_{SP2} \cdot h_{SP2} = 25.819\ MW
\end{equation*}
A questa potenza devo sottrarre quella necessaria alle pompe, ricavabile analogamente.
\begin{equation*}
    P_{p_estrazione} = \dot m_1 \cdot \left( h_{1'} - h_1 \right) = 13.4\ kW \\
    P_{p_alimento} = \dot m_2 \cdot \left( h_2 - h_{2'} \right) = 109\ kW 
\end{equation*}
La \textbf{potenza limite} è dunque pari a 25.697 MW.

La potenza termica trasferita al fluido è pari alla portata per il salto di entalpia a monte e a valle del generatore di vapore
\begin{equation*}
    \dot Q_1 = \dot m_2 \cdot (h_3-h_2) = 62.709\ MW
\end{equation*}
Il rendimento limite è dunque pari al 40.98\%

Per il confronto del rendimento limite con il rendimento di Carnot è necessario calcolare le temperature medie di adduzione e sottrazione del calore
utilizzando nuovamente la \eqref{eq:T_ds}.

La temperatura media di adduzione del calore è pari a
\begin{equation*}
    T_{ma} = \frac{h_3 - h_2}{s_3 - s_2} = 517.4\ K = 244.2\ \text{°}C
\end{equation*}
La temperatura media di sottrazione del calore vale invece:
\begin{equation*}
    T_{ms} = \frac{h_4-h_1}{s_4 - s_1} = 302.9\ K = 29.7\ \text{°}C
\end{equation*}
Utilizzando la \eqref{eq:rendimento_carnot} troviamo il rendimento di Carnot pari a 41.46\%.

\subsection{Rendimento interno di impianto}
Ricaviamo ancora una volta la potenza reale erogata dalla turbina, riferendoci questa volta ai dati termodinamici reali.
\begin{equation*}
    P_T = \dot m_3 \cdot h_3 - \dot m_{SP2} \cdot h_{SP2} - \dot m_4 \cdot h_4 = 20.531\ MW
\end{equation*}
Calcolo ora la potenza necessaria ai meccanismi ausiliari, posso trascurare la potenza necessaria alla pompa di estrazione, non conosco in ogni caso lo stato termodinamico
reale successivo alla pompa.
\begin{equation*}
    P_{p_alimento} = \dot m_2 \cdot \left( h_2 - h_{2'} \right) = 250\ kW 
\end{equation*}
La potenza netta reale sarà circa 20.280 MW.

Il rendimento interno vale dunque 78.92\%.

\subsection{Rendimento reale}
È necessario il calcolo della potenza termica trasferita nell'impianto reale per poter valutare il rendimento reale:
\begin{equation*}
    \dot Q_1 = \dot m_2 \cdot (h_{3'} - h_2) = 63.130\ MW
\end{equation*}
Il rendimento reale è dunque 32.12\%.

\subsection{Rendimento meccanico}
Si assume che l'alternatore sia identico all'impianto precedente e così anche il suo rendimento pari a 97.5\%.
La $P_{ua}$ sarà pari quindi a 20.195 MW. Il rendimento meccanico quindi vale 99.58\%.

\subsection{Rendimento adiabatico di espansione}
Valutata nella sezione \ref{subsec:rendimento_limite_2} la potenza ideale disponibile in turbina, è possibile calcolare il rendimento adiabatico di espansione pari a
79.90\%.

\subsection{Verifica dei rendimenti}
\begin{align*}
    \eta_g \stackrel{sch}{=} \eta_b \cdot \eta_l \cdot \eta_{ii} \cdot \eta_m
    = \frac{\dot{Q_1}}{\dot{m}_c H_i}\cdot \frac{P_l}{\dot{Q_1}} \cdot \frac{P_r}{P_l}\cdot \frac{P_{ua}}{P_r} = \\
    =  0.8977 \cdot 0.4098 \cdot 0.7892 \cdot 0.9958 = 28.91 \%
\end{align*}
