\documentclass[a4paper,12pt]{article}
\usepackage[T1]{fontenc}
\usepackage[utf8]{inputenc}
\usepackage{mathtools}
\usepackage[italian]{babel}
%\usepackage{graphicx}
\usepackage{float}
\usepackage{textcomp}
\usepackage{amsmath}

\title{Esercizio 3}
\author{Olivieri Daniele}
\date{}

\begin{document}
\maketitle

Analizzare un impianto con turbina a gas di derivazione aereonautica LM6000 valutando i cicli termodinamici ideale e reale stimando i vari rendimenti.
\section{Analisi etichetta}
\label{sec:analisi_etichetta}
I dati utili all'analisi termodinamica sono i seguenti:
\begin{center}
    \begin{tabular}{cccccc}
         Potenza   & Heat  & Efficienza   & Rapp.  & Portata    &  Temp. \\
                   &rate  &              & di comp. &    &        all'uscita \\
            (kW)   &  (kJ/kWh) & \%    &    $\beta$  &   kg/s      &  °C \\ \hline
        43100    &  8620    &  41.7   &  29.5  &     125.0        &  450
    \end{tabular}
\end{center}
mentre per le condizioni dell'aria in ingresso alla turbina si può facilmente supporre che la temperatura sia pari a 15°C mentre la pressione 1 atm
(atmosfera standard ICAO).



\end{document}
