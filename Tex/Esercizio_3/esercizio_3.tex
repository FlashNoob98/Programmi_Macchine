\documentclass[a4paper,12pt]{article}
\usepackage[T1]{fontenc}
\usepackage[utf8]{inputenc}
\usepackage{mathtools}
\usepackage[italian]{babel}
%\usepackage{graphicx}
\usepackage{float}
\usepackage{textcomp}
\usepackage{amsmath}

\title{Esercizio 3}
\author{Olivieri Daniele}
\date{}

\begin{document}
\maketitle

Analizzare un impianto con turbina a gas di derivazione aereonautica LM6000 valutando i cicli termodinamici ideale e reale stimando i vari rendimenti.
\section{Analisi etichetta}
\label{sec:analisi_etichetta}
I dati utili all'analisi termodinamica sono i seguenti:
\begin{center}
    \begin{tabular}{cccccc}
         Potenza   & Heat  & Efficienza   & Rapp.  & Portata    &  Temp. \\
                   &rate  &              & di comp. &    &        all'uscita \\
            (kW)   &  (kJ/kWh) & \%    &    $\beta$  &   kg/s      &  °C \\ \hline
        43100    &  8620    &  41.7   &  29.5  &     125.0        &  450
    \end{tabular}
\end{center}
Per le condizioni dell'aria in ingresso alla turbina si può facilmente supporre che la temperatura sia pari a 15°C mentre la pressione 1 atm
(atmosfera standard ICAO).

\section{Ciclo reale}
\label{sec:ciclo_reale}
Iniziamo l'analisi termodinamica dell'impianto a partire dagli stati del ciclo reale.
Calcoliamo il volume specifico $v_1$ utilizzando l'equazione di stato dei gas perfetti:
\begin{equation}
    \label{eq:gas_perfetti}
    pv = RT
\end{equation}
Per lo stato 2 fissiamo invece il valore di pressione $p_2$ uguale a $\beta$
Il calcolo della potenza termica da fornire in camera di combustione si esegue attraverso l'analisi dell'\textit{Heat Rate} espresso in kJ/kWh.
Tale grandezza va divisa per 3600 per renderla adimensionale, ottenendo un valore di 2.395, se la potenza elettrica generata è di 43.100 MW, 
la potenza termica necessaria al funzionamento dell'impianto sarà 103.200 MW.
Tale valore è pari alla portata massica di combustibile per il suo potere calorifico inferiore. Utilizzando gas naturale esso è pari a 47.7 MJ/kg e la portata di combustibile
è di 2.16 kg/s.
Supponendo un rendimento di combustione $\eta_b$ pari al 99\%, la potenza termica effettivamente trasferita all'aria in camera di combustione è pari a
102.00 MW. Da questo valore possiamo ricavare il salto entalpico subito dal gas e quindi la differenza di temperatura tra lo stato 2 e 3 ipotizzando costante il $C_p$
dell'aria a 1.005 kJ/(kg K).
$T_3 -T_2$ sarà uguale a 811.9 K.

Nell'ipotesi in cui i rendimenti politropici di compressione ed espansione siano uguali, si possono dunque ricavare le temperature $T_2$ e $T_3$ impostando il seguente sistema
($\lambda = (k-1)/k;\ k=1.4$)
$$
\begin{cases}
    T_2 = T_1\cdot\beta^{\frac{\lambda}{\eta_p}} \\
    T_3 = T_4\cdot\beta^{\frac{\lambda}{\eta_p}}\\
    T_3-T_2 = 811.9
\end{cases}
$$
Il rendimento politropico vale 0.918 e le due temperature valgono rispettivamente 551.14 °C e 1795 °C.
Si può quindi completare la tabella degli stati reali.
\begin{center}
    \begin{tabular}{c|c|c|c}
            &p(atm) &T(°C)  &v$(m^3/kg)$     \\ \hline
        1   &    1  & 15    & 0.827   \\
        2   &   29.5&553    & 0.080   \\
        3   &   29.5&1800   & 0.201   \\
        4   &   1   & 450   & 2.07
    \end{tabular}
\end{center}

\section{Ciclo limite}
\label{sec:ciclo_limite}
Segue l'analisi del ciclo Joule limite considerando il compressore e la turbina isoentropici, e la combustione isobara.
Per lo stato 2 fissiamo invece il valore di pressione $p_2$ uguale a $\beta$ e calcoliamo il volume specifico raggiunto dopo la compressione isoentropica.
Consideriamo inoltre lo stato 3 calcolato nella sezione \ref{sec:ciclo_reale} come punto di partenza per il calcolo delle condizioni all'uscita (stato 4).
\begin{equation*}
    v_2 = v_1/(\beta^{1/k})
\end{equation*}
e nuovamente $T_2$ mediante la \eqref{eq:gas_perfetti}.
\begin{center}
    \begin{tabular}{c|c|c|c}
            &p(atm) &T(°C)  &v$(m^3/kg)$     \\ \hline
        1   &    1  & 15    & 0.827   \\
        2   &   29.5& 484   & 0.0737  \\
        3   &   29.5&1800   & 0.201   \\
    \end{tabular}
\end{center}


\end{document}
