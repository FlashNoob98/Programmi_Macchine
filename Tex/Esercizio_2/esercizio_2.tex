\documentclass[a4paper,12pt]{article}
\usepackage[T1]{fontenc}
\usepackage[utf8]{inputenc}
\usepackage{mathtools}
\usepackage[italian]{babel}
%\usepackage{graphicx}
\usepackage{float}
\usepackage{textcomp}
\usepackage{amsmath}

\title{Esercitazione 2}
\author{Olivieri Daniele}
\date{}

\begin{document}
\maketitle
% A partire dai dati sperimentali riportati nei tre schemi d'impianto calcolare
% \begin{itemize}
%     \item I rendimenti:
%     \begin{itemize}
%         \item globale elettrico
%         \item globale
%         \item di combustione
%         \item limite
%         \item interno d'impianto
%         \item reale
%         \item meccanico
%         \item adiabatico di espansione
%     \end{itemize}
%     \item Le temperature medie di adduzione e sottrazione del calore
% \end{itemize}
% Confrontare i risultati.

\section{Definizione dei rendimenti}
A partire dai dati sperimentali riportati nei tre schemi d'impianto calcolare i seguenti rendienti:
\begin{itemize}
    \item Rendimento globale elettrico
    \begin{equation}
        \label{eq:rendimento_elettrico}
        \eta_{g_{el}} \stackrel{def}{=} \frac{P_{el}}{\dot m_c \cdot H_i}
    \end{equation}
    
    \item Rendimento globale
    \begin{equation}
        \label{eq:rendimento_globale}
        \eta_g \stackrel{def}{=} \frac{P_{ua}}{\dot{m}_c \cdot H_i}
   \end{equation}

   \item Rendimento di combustione
   \begin{equation}
       \label{eq:rendimento_combustione}
       \eta_C \stackrel{def}{=} \frac{\dot{Q_1}}{\dot{m}_c \cdot H_i}
   \end{equation}

   \item Rendimento limite
   \begin{equation}
       \label{eq:redimento_limite}
       \eta_l \stackrel{def}{=} \frac{P_l}{\dot{Q_1}}
   \end{equation}

   \item Rendimento interno d'impianto
   \begin{equation}
        \label{eq:rendimento_interno}
       \eta_{ii} \stackrel{def}{=} \frac{P_r}{P_l}
   \end{equation}

   \item Rendimento reale
   \begin{equation}
       \label{eq:rendimento_reale}
       \eta_r \stackrel{def}{=} \eta_l \cdot \eta_{ii} = \frac{P_l}{\dot{Q_1}} \cdot \frac{P_r}{P_l} = \frac{P_r}{\dot{Q_1}}
   \end{equation}

   \item Rendimento meccanico
   \begin{equation}
       \label{eq:rendimento_meccanico}
       \eta_m \stackrel{def}{=} \frac{P_{ua}}{P_r}
   \end{equation}


   \item Rendimento adiabatico di espansione
   \begin{equation}
        \label{eq:rendimento_adiabatico}
        \eta_{ad_e} \stackrel{def}{=} \frac{L_{is}}{L_r} = \frac{1-1/\beta^{\frac{m-1}{m}}}
        {1-1/\beta^{\frac{k-1}{k}}}
   \end{equation}
   oppure utilizzando il rendimento politropico
   \begin{equation}
        \label{eq:rendimento_adiabatico_con_rendimento_politropico}
        \eta_{ad_e} = \frac{1-1/\beta^{\frac{k-1}{k}\eta_{pe}}}
        {1-1/\beta^{\frac{k-1}{k}}}
        %uso \displaystyle per forzare l'integrale grande
   \end{equation}
\end{itemize}
Valutare inoltre le temperature medie di adduzione e sottrazione del calore.

\section{Impianto minimo}
\label{sec:primo_impianto}
\subsection{Rendimento globale elettrico}
\label{subsec:rendimento_globale_elettrico}
Il primo schema presenta i componenti minimi necessari alla realizzazione di un impianto con turbina a vapore,
è possibile valutare rapidamente il rendimento elettrico dell'impianto dopo aver trasformato la portata massica di combustibile da 
tonnellate/ora in kilogrammi/secondo, ossia dividere per 3.6, il risultato sarà 1524.4 kg/s di gas naturale immessi nel generatore di vapore.
Il potere calorifico inferiore, ossia la massima quantità di calore ottenibile da una combustione completa del gas naturale è di 47.7 MJ/kg
Il \textbf{rendimento globale elettrico} sarà dunque pari a 27.7 \%.

\subsection{Rendimento globale}
\label{subsec:rendimento_globale}
In prima approssimazione il rendimento globale può essere confuso con quello elettrico, in questo caso infatti la turbina ha una velocità
di 3000 giri al minuto ossia 50 Hz, non è necessario collegare un riduttore meccanico tra la turbina e l'alternatore.
Il rendimento dell'alternatore, per una potenza di circa 20 MW è stimabile da valori tabellati, pari a circa il 97.5 \%.
La potenza utile all'asse $P_{ua}$ sarà dunque il rapporto della potenza elettrica generata per il rendimento dell`alternatore, ossia
20.625 MW, fornendo un rendimento \textbf{globale} di impianto pari a 28.37 \%. 

\subsection{Rendimento di combustione}
\label{subsec:rendimento_combustione}
Definito nella \eqref{eq:rendimento_combustione} il rendimento di combustione esprime il rapporto tra la quantità di calore
fornita al fluido e quella potenzialmente ottenibile dal combustibile.

La potenza termica trasferita al fluido è valutabile come la portata di fluido per la variazione di entalpia $\dot m\cdot \Delta h$
cioè 3276 kJ/kg per 19.89 kg/s ossia 65.156 MW di potenza termica trasferita.

La potenza termica potenziale è pari invece alla portata di combustibile per il suo potere calorifico inferiore:
1524 kg/s per 47.7 MJ/kg ossia 72.716 MW.

Il \textbf{rendimento di combustione} è dunque pari a 89.60\%

\subsection{Rendimento limite}
\label{subsec:rendimento_limite}
Per poter calcolare il rendimento limite definito nella \eqref{eq:redimento_limite} è necessaria un'analisi
più approfondita dell'impianto che permetta di valutare il ciclo termodinamico limite, ovvero quello che avverrebbe
con processi di scambio termico isobari ed espansione/compressione adiabatici isoentropici. 

Notando che i valori di entalpia riportati nello schema sono espressi in kcal/kg e che 1 kcal è pari a 4.186 kJ,
bisogna moltiplicare per questo fattore la differenza di entalpia.
Utilizziamo come punti di riferimento i valori reali misurati nello stato termodinamico 1 e 3 ossia a monte delle macchine
adiabatiche, possiamo raccogliere i dati in una tabella
(la pressione è stata trasformata da \textit{ata} ad \textit{atm} mentre l'entalpia da \textit{kcal} a \textit{kJ}):
\begin{center}
    \begin{tabular}{r|c|c|c}
        stato    & $p\ (atm)$ & $h\ (kJ/kg)$ & $T\ (\text{°}C) $\\ \hline
        1   &        0.0419 &          125.6   &           30     \\ \hline
        3   &        59.05  &           3381   &           485
    \end{tabular}
\end{center}
Confrontanto il valore di entalpia con quello delle tabelle dell'acqua in condizioni di saturazione possiamo affermare
che l'acqua si trova nella condizione di liquido saturo, possiamo quindi leggere la pressione dalla stessa tabella.
Consideriamo unica la compressione eseguita dalle due pompe.

Riportiamo questi valori in una tabella che comprenda anche l'entropia ed il titolo per il calcolo degli altri 2 stati termodinamici,
lo stato 1 come precedentemente detto è in condizioni di liquido saturo mentre lo stato 3 è nella condizione di vapore surriscaldato
nel momento in cui la temperatura è maggiore della temperatura critica dell'acqua (373.9 °C) e la pressione è minore di quella critica
(217.7 atm).
Ricordando le ipotesi precedenti possiamo fissare pressione ed entropia per gli stati 2 e 4.
I valori di temperatura, titolo ed entalpia per lo stato 2 e 4 sono stimabili mediante l'uso del diagramma di Mollier,
tabelle o software appositi.
\begin{center}
    \begin{tabular}{r|c|c|c|c|c}
        stato    & $p\ (atm)$ & $T\ (\text{°}C) $&$X\ (\%$)& $h\ (\frac{kJ}{kg})$  & $s\ (\frac{kJ}{kg\cdot K})$\\ \hline
        1   &        0.0419 &            30   &    0      & 125.6    &0.4368 \\ \hline
        2   &        59.05    &          30.137&liq. sott.& 131.7    &0.4368 \\ \hline
        3   &        59.05  &           485   & vap. surr.& 3381     &6.837  \\ \hline
        4   &        0.0419       &      29.98&     79.85 &  2066    &6.837
    \end{tabular}
\end{center}

Siamo ora in possesso di tutti i dati necessari al calcolo della potenza limite ricavata dal ciclo termodinamico ideale.
La potenza meccanica ricavata dall'espansione in turbina sarà pari ancora una volta al prodotto della portata massica per il salto
entalpico del fluido, $\dot m \cdot (h_3 - h_4) $ ossia 19.65 kg/s per 1315 kJ/kg, una potenza di 25.840 MW.
A questa va sottratta la potenza meccanica necessaria alle pompe (in questo caso racchiusa in un'unica trasformazione)
pari a $\dot m \cdot (h_2 - h_1)$ = 121 kW.

La \textbf{potenza meccanica limite} è quindi 25.719 MW, mentre la potenza termica fornita al fluido è pari al salto di entalpia tra 
lo stato 3 e 2, 64.258 MW.
Il \textbf{rendimento limite} è del 40.02 \%

Il rendimento limite è pari al rendimento del ciclo ideale di Carnot che avviene tra le stesse temmperature medie di adduzione e sottrazione
del calore, temperature stimabili dalla seguente:
\begin{equation}
    \label{eq:T_ds}
    Q = T\Delta s
\end{equation}
La temperatura media di adduzione del calore è pari a:
\begin{equation*}
    T_{m_a} = \frac{h_3-h_2}{s_3 - s_2} = 507.68\ K = 234.53 \text{°C}
\end{equation*}
mentre quella di sottrazione è pari a:
\begin{equation*}
    T_{m_s} = \frac{h_4 - h_1}{s_4 - s_1} = 303.17\ K = 30.02 \text{°C}
\end{equation*}
Il rendimento del ciclo di Carnot è quindi definito come
\begin{equation}
    \eta_{Carnot} \stackrel{def}{=} 1 - \frac{T_{m_s}}{T_{m_a}}
\end{equation}
e utilizzando le temperature in Kelvin esso vale 0.402 ossia 40.2 \%.
%Il rendimento limite raggiunge il rendimento di Carnot se trascuriamo la potenza meccanica necessaria alla compressione del fluido.

\subsection{Rendimento interno di impianto}
Per il calcolo del rendimento interno di impianto è necessario valutare la potenza ottenuta dalla trasformazione reale,
valutabile come la portata di fluido per il salto di entalpia al quale va sottratta la potenza meccanica necessaria al funzionamento
dei dispositivi ausiliari (pompe).

È presente una fuga di vapore in turbina quindi per una stima più precisa consideriamo la portata massica più bassa pari
a 70.72 t/h, pari a 19.65 kg/s. La variazione di entalpia è pari a 254.9 kcal/kg o 1067 kJ/kg.
La potenza ottenuta dalla turbina è dunque 20.960 MW.

Nell'impianto sono presenti 2 pompe, una di estrazione dal condensatore e l'altra di alimento verso il generatore
di vapore, entrambi i lavori sono stimabili attraverso la variazione di entalpia subita dal fluido:
la pompa di \textbf{estrazione} produce un aumento di entalpia pari a 0.7 kcal/kg o 2.93 kJ/kg con una portata massica di
71.6 t/h o 19.89 kg/s, la potenza meccanica necessaria sarà dunque 58.3 kW, una modesta quantità rispetto a quella prodotta dalla turbina.
La pompa di \textbf{alimento} produce una variazione di entalpia pari a 2.11 kcal/kg o 8.833 kJ/kg, la portata massica è identica
alla precedente, la potenza necessaria è 175.69 kW, maggiore della pompa precedente ma comunque molto piccola rispetto alla potenza
ricavata dalla turbina, è proprio questo il grande vantaggio degli impianti a vapore rispetto agli impianti a gas dove la potenza
necessaria alla compressione del gas non è affatto trascurabile.
La potenza meccanica \textbf{reale} è dunque 20.726 MW.

Definito nella \eqref{eq:rendimento_interno} il rendimento \textbf{interno} esprime il rapporto tra la potenza ricavata dall'impianto reale
e quella massima ricavabile dallo stesso impianto se fosse ideale, esprime cioè la bontà costruttiva dell'impianto rispetto al limite ideale.
La potenza meccanica ideale è stata calcolata nella sezione \ref{subsec:rendimento_limite} e valutando il rendimento si ottiene un valore di 
80.59\%.

\subsection{Rendimento reale}
Il rendimento reale definito nella \eqref{eq:rendimento_reale} è pari al prodotto tra i rendimenti interno e limite, vale dunque 32.25 \%


\end{document}
