\documentclass[a4paper,12pt]{article}
\usepackage[T1]{fontenc}
\usepackage[utf8]{inputenc}
\usepackage{mathtools}
\usepackage[italian]{babel}
%\usepackage{graphicx}
\usepackage{float}
\usepackage{textcomp}
\usepackage{amsmath}

\title{Esercitazione 2}
\author{Olivieri Daniele}
\date{}

\begin{document}
\maketitle
% A partire dai dati sperimentali riportati nei tre schemi d'impianto calcolare
% \begin{itemize}
%     \item I rendimenti:
%     \begin{itemize}
%         \item globale elettrico
%         \item globale
%         \item di combustione
%         \item limite
%         \item interno d'impianto
%         \item reale
%         \item meccanico
%         \item adiabatico di espansione
%     \end{itemize}
%     \item Le temperature medie di adduzione e sottrazione del calore
% \end{itemize}
% Confrontare i risultati.

\section{Definizione dei rendimenti}
A partire dai dati sperimentali riportati nei tre schemi d'impianto calcolare i seguenti rendienti:
\begin{itemize}
    \item Rendimento globale elettrico
    \begin{equation}
        \label{eq:rendimento_elettrico}
        \eta_{g_{el}} \stackrel{def}{=} \frac{P_{el}}{\dot m_c \cdot H_i}
    \end{equation}
    
    \item Rendimento globale
    \begin{equation}
        \label{eq:rendimento_globale}
        \eta_g \stackrel{def}{=} \frac{P_{ua}}{\dot{m}_c \cdot H_i}
   \end{equation}

   \item Rendimento di combustione
   \begin{equation}
       \label{eq:rendimento_combustione}
       \eta_b \stackrel{def}{=} \frac{\dot{Q_1}}{\dot{m}_c \cdot H_i}
   \end{equation}

   \item Rendimento limite
   \begin{equation}
       \label{eq:redimento_limite}
       \eta_l \stackrel{def}{=} \frac{P_l}{\dot{Q_1}}
   \end{equation}

   \item Rendimento interno d'impianto
   \begin{equation}
        \label{eq:rendimento_interno}
       \eta_{ii} \stackrel{def}{=} \frac{P_r}{P_l}
   \end{equation}

   \item Rendimento reale
   \begin{equation}
       \label{eq:rendimento_reale}
       \eta_r \stackrel{def}{=} \eta_l \cdot \eta_{ii} = \frac{P_l}{\dot{Q_1}} \cdot \frac{P_r}{P_l} = \frac{P_r}{\dot{Q_1}}
   \end{equation}

   \item Rendimento meccanico
   \begin{equation}
       \label{eq:rendimento_meccanico}
       \eta_m \stackrel{def}{=} \frac{P_{ua}}{P_r}
   \end{equation}


   \item Rendimento adiabatico di espansione
   \begin{equation}
        \label{eq:rendimento_adiabatico}
        \eta_{ad_e} \stackrel{def}{=} \frac{P_r}{P_{is}}
    \end{equation}
%          = \frac{1-1/\beta^{\frac{m-1}{m}}}
%         {1-1/\beta^{\frac{k-1}{k}}}
%    oppure utilizzando il rendimento politropico
%    \begin{equation}
%         \label{eq:rendimento_adiabatico_con_rendimento_politropico}
%         \eta_{ad_e} = \frac{1-1/\beta^{\frac{k-1}{k}\eta_{pe}}}
%         {1-1/\beta^{\frac{k-1}{k}}}
%         %uso \displaystyle per forzare l'integrale grande
%    \end{equation}
\end{itemize}
Valutare inoltre le temperature medie di adduzione e sottrazione del calore.

\section{Impianto minimo da 20.11 MW}
\label{sec:primo_impianto}
\subsection{Rendimento globale elettrico}
\label{subsec:rendimento_globale_elettrico}
Il primo schema presenta i componenti minimi necessari alla realizzazione di un impianto con turbina a vapore;
è possibile valutare rapidamente il rendimento elettrico dell'impianto dopo aver trasformato la portata massica di combustibile da 
tonnellate/ora in kilogrammi/secondo, ossia dividere per 3.6. Il risultato sarà 1524.4 kg/s di gas naturale immessi nel generatore di vapore.
Il potere calorifico inferiore, ossia la massima quantità di calore ottenibile da una combustione completa del gas naturale è di 47.7 MJ/kg.
Il \textbf{rendimento globale elettrico} sarà dunque pari a 27.7 \%.

\subsection{Rendimento globale}
\label{subsec:rendimento_globale}
In prima approssimazione il rendimento globale può essere confuso con quello elettrico. In questo caso, infatti, la turbina ha una velocità
di 3000 giri al minuto ossia 50 Hz, quindi non è necessario collegare un riduttore meccanico tra la turbina e l'alternatore.
Il rendimento dell'alternatore, per una potenza di circa 20 MW, è stimabile da valori tabellati ed è pari a circa il 97.5 \%.
La potenza \textbf{utile all'asse} $P_{ua}$ sarà dunque il rapporto della potenza elettrica generata per il rendimento dell'alternatore, ossia
20.625 MW, fornendo un rendimento \textbf{globale} di impianto pari a 28.37 \%. 

\subsection{Rendimento di combustione}
\label{subsec:rendimento_combustione}
Definito nella \eqref{eq:rendimento_combustione} il rendimento di combustione esprime il rapporto tra la quantità di calore
fornita al fluido e quella potenzialmente ottenibile dal combustibile.

La potenza termica trasferita al fluido è valutabile come la portata di fluido per la variazione di entalpia $\dot m\cdot \Delta h$
cioè 3276 kJ/kg per 19.89 kg/s ossia 65.156 MW di potenza termica trasferita.

La potenza termica potenziale è pari invece alla portata di combustibile per il suo potere calorifico inferiore:
1524 kg/s per 47.7 MJ/kg ossia 72.716 MW.

Il \textbf{rendimento di combustione} è dunque pari a 89.60\%

\subsection{Rendimento limite}
\label{subsec:rendimento_limite}
Per poter calcolare il rendimento limite definito nella \eqref{eq:redimento_limite} è necessaria un'analisi
più approfondita dell'impianto che permetta di valutare il ciclo termodinamico limite, ovvero quello che avverrebbe
con processi di scambio termico isobari ed espansione/compressione adiabatici isoentropici. 

Notando che i valori di entalpia riportati nello schema sono espressi in kcal/kg e che 1 kcal è pari a 4.186 kJ,
bisogna moltiplicare per questo fattore la differenza di entalpia.
Utilizzando come punti di riferimento i valori reali misurati nello stato termodinamico 1 e 3 ossia a monte delle macchine
adiabatiche, è possibile raccogliere i dati in una tabella 
(la pressione è stata trasformata da \textit{ata} ad \textit{atm} mentre l'entalpia da \textit{kcal} a \textit{kJ}):
\begin{center}
    \begin{tabular}{r|c|c|c}
        stato    & $p\ (atm)$ & $h\ (kJ/kg)$ & $T\ (\text{°}C) $\\ \hline
        1   &        0.0419 &          125.6   &           30     \\ \hline
        3   &        59.05  &           3381   &           485
    \end{tabular}
\end{center}
Confrontanto il valore di entalpia con quello delle tabelle dell'acqua in condizioni di saturazione possiamo affermare
che l'acqua si trova nella condizione di liquido saturo, possiamo quindi leggere la pressione dalla stessa tabella.
Consideriamo unica la compressione eseguita dalle due pompe.

Riportiamo questi valori in una tabella che comprenda anche l'entropia ed il titolo per il calcolo degli altri 2 stati termodinamici,
lo stato 1, come precedentemente detto, è in condizioni di liquido saturo mentre lo stato 3 è nella condizione di vapore surriscaldato
poichè la temperatura è maggiore della temperatura critica dell'acqua (373.9 °C) e la pressione è minore di quella critica
(217.7 atm).
Ricordando le ipotesi precedenti possiamo fissare pressione ed entropia per gli stati 2 e 4.
I valori di temperatura, titolo ed entalpia per lo stato 2 e 4 sono stimabili mediante l'uso del diagramma di Mollier,
tabelle o software appositi.
\begin{center}
    \begin{tabular}{r|c|c|c|c|c}
        stato    & $p\ (atm)$ & $T\ (\text{°}C) $&$X\ (\%$)& $h\ (\frac{kJ}{kg})$  & $s\ (\frac{kJ}{kg\cdot K})$\\ \hline
        1   &        0.0419 &            30   &    0      & 125.6    &0.4368 \\ \hline
        2   &        59.05    &          30.137&liq. sott.& 131.7    &0.4368 \\ \hline
        3   &        59.05   &           485  & vap. surr.& 3381     &6.837  \\ \hline
        4   &        0.0419       &      29.98&     79.85 & 2066     &6.837
    \end{tabular}
\end{center}

Siamo ora in possesso di tutti i dati necessari al calcolo della potenza limite ricavata dal ciclo termodinamico ideale.
La potenza meccanica ricavata dall'espansione in turbina sarà pari ancora una volta al prodotto della portata massica per il salto
entalpico del fluido, $\dot m \cdot (h_3 - h_4) $ ossia 19.65 kg/s per 1315 kJ/kg, una potenza di 25.840 MW.
A questa va sottratta la potenza meccanica necessaria alle pompe (in questo caso racchiusa in un'unica trasformazione)
pari a $\dot m \cdot (h_2 - h_1)$ = 121 kW.

La \textbf{potenza meccanica limite} è quindi 25.719 MW, mentre la potenza termica fornita al fluido è pari al salto di entalpia tra 
lo stato 3 e 2, 64.258 MW.
Il \textbf{rendimento limite} è del 40.02 \%

Il rendimento limite è pari al rendimento del ciclo ideale di Carnot che avverrebbe tra le stesse temmperature medie di adduzione e sottrazione
del calore, temperature stimabili dalla seguente:
\begin{equation}
    \label{eq:T_ds}
    Q = T\Delta s
\end{equation}
La temperatura media di adduzione del calore è pari a:
\begin{equation*}
    T_{m_a} = \frac{h_3-h_2}{s_3 - s_2} = 507.68\ K = 234.53 \text{°C}
\end{equation*}
mentre quella di sottrazione è pari a:
\begin{equation*}
    T_{m_s} = \frac{h_4 - h_1}{s_4 - s_1} = 303.17\ K = 30.02 \text{°C}
\end{equation*}
Il rendimento del ciclo di Carnot è quindi definito come
\begin{equation}
    \eta_{Carnot} \stackrel{def}{=} 1 - \frac{T_{m_s}}{T_{m_a}}
\end{equation}
e utilizzando le temperature in Kelvin esso vale 0.402 ossia 40.2 \%.
%Il rendimento limite raggiunge il rendimento di Carnot se trascuriamo la potenza meccanica necessaria alla compressione del fluido.

\subsection{Rendimento interno di impianto}
\label{subsec:rendimento_interno}
Per il calcolo del rendimento interno di impianto è necessario valutare la potenza ottenuta dalla trasformazione reale,
valutabile come la portata di fluido per il salto di entalpia al quale va sottratta la potenza meccanica necessaria al funzionamento
dei dispositivi ausiliari (pompe).

È presente una fuga di vapore in turbina quindi per una stima più precisa consideriamo la portata massica più bassa pari
a 70.72 t/h, pari a 19.65 kg/s. La variazione di entalpia è pari a 254.9 kcal/kg o 1067 kJ/kg.
La potenza ottenuta dalla turbina è dunque 20.960 MW.

Nell'impianto sono presenti 2 pompe, una di estrazione dal condensatore e l'altra di alimento verso il generatore
di vapore; entrambi i lavori sono stimabili attraverso la variazione di entalpia subita dal fluido:
la pompa di \textbf{estrazione} produce un aumento di entalpia pari a 0.7 kcal/kg o 2.93 kJ/kg con una portata massica di
71.6 t/h o 19.89 kg/s, la potenza meccanica necessaria sarà dunque 58.3 kW, una modesta quantità rispetto a quella prodotta dalla turbina.
La pompa di \textbf{alimento} produce una variazione di entalpia pari a 2.11 kcal/kg o 8.833 kJ/kg, la portata massica è identica
alla precedente, la potenza necessaria è 175.69 kW, maggiore della pompa precedente ma comunque molto piccola rispetto alla potenza
ricavata dalla turbina.

È proprio questo il grande vantaggio degli impianti a vapore rispetto agli impianti a gas dove la potenza
necessaria alla compressione del gas non è affatto trascurabile.
La potenza meccanica \textbf{reale} è dunque 20.726 MW.

Definito nella \eqref{eq:rendimento_interno} il rendimento \textbf{interno} esprime il rapporto tra la potenza ricavata dall'impianto reale
e quella massima ricavabile dallo stesso impianto se fosse ideale, esprime cioè la bontà costruttiva dell'impianto rispetto al limite ideale.
La potenza meccanica ideale è stata calcolata nella sezione \ref{subsec:rendimento_limite} e valutando il rendimento si ottiene un valore di 
80.59\%.

\subsection{Rendimento reale}
Il rendimento reale definito nella \eqref{eq:rendimento_reale} è pari al prodotto tra i rendimenti interno e limite, vale dunque 32.25 \%

\subsection{Rendimento meccanico}
Definito nella \eqref{eq:rendimento_meccanico} esprime le perdite meccaniche ed è pari a 0.995

\subsection{Rendimento adiabatico di espansione}
Il rendimento adiabatico di espansione permette di stimare le irreversibilità introdotte dall'espansione in turbina come definito dalla \eqref{eq:rendimento_adiabatico}.
La potenza meccanica \textbf{limite} è stata calcolata nella sezione \ref{subsec:rendimento_limite} e vale 25.840 MW (ignorando la spesa necessaria alla compressione),
mentre quella \textbf{reale} nella sezione \ref{subsec:rendimento_interno} e vale 20.960 MW.
Il rendimento \textbf{adiabatico} di espansione vale dunque 81.11\%.

\subsection{Considerazioni finali}
Possiamo verificare il calcolo dei rendimenti effettuati mediante la seguente schematizzazione del rendimeto globale:
\begin{equation}
    \eta_g \stackrel{sch}{=} \eta_b \cdot \eta_l \cdot \eta_{ii} \cdot \eta_m
    = \frac{\dot{Q_1}}{\dot{m}_c H_i}\cdot \frac{P_l}{\dot{Q_1}} \cdot \frac{P_r}{P_l}\cdot \frac{P_{ua}}{P_r}
\end{equation}
e ricapitolando i rendimenti essi valgono:
\begin{itemize}
    \item Rendimento di combustione: 0.8960
    \item Rendimento limite: 0.4002
    \item Rendimento interno: 0.8059
    \item Rendimento meccanico: 0.995
\end{itemize}
Il loro prodotto vale 0.2875, ossia 28.75\% molto prossimo al rendimento globale calcolato all'inizio della trattazione (\ref{subsec:rendimento_globale}).

\section{Secondo impianto da 19.69 MW}
Il secondo impianto prevede rispetto al primo uno \textbf{spillamento} dalla turbina, ossia una sottrazione di vapore caldo da uno degli stadi della turbina, 
utile al preriscaldamento del fluido di lavoro prima dell'ingresso nel generatore di vapore.
La potenza elettrica generata è quindi diminuita ma è diminuita anche la portata di combustibile, è necessario quindi rieseguire il calcolo dei rendimenti.
\subsection{Rendimento globale}
Il rendimento globale elettrico calcolabile con la \eqref{eq:rendimento_elettrico} vale 27.66\%.

Considerando il rendimento dell'alternatore identico al caso precedente, il rendimento globale d'impianto è pari a 28.36\%
\subsection{Rendimento di combustione}
La potenza termica trasferita al fluido nel generatore di vapore è valutabile con il suo salto entalpico e pari a 63.10 MW.
Il calore prodotto dal combustibile, calcolato in precedenza per i rendimenti globali è pari a 70.29 MW.
Il rendimento di combustione è quindi pari al 89.77\%.

\subsection{Rendimento limite}
\label{subsec:rendimento_limite_2}
Eseguiamo ora l'analisi del ciclo limite considerando come valori di riferimento quelli dello stato 1 e 3.
\begin{center}
    \begin{tabular}{r|c|c|c}
        stato    & $p\ (atm)$ & $h\ (kJ/kg)$ & $T\ (\text{°}C) $\\ \hline
        1   &        0.0419 &          125.6   &           30     \\ \hline
        3   &        59.05  &           3381   &           485
    \end{tabular}
\end{center}
Considero SP2 lo stato termodinamico successivo allo spillamento dalla turbina, con una portata massica di 0.725 kg/s. Ricalcolo la tabella degli stati.
Non è possibile calcolare immediatamente lo stato 2' dato che è successivo ad un mescolamento adiabatico con il vapore spillato dalla turbina, calcolo quindi prima
lo stato 4 in uscita dalla turbina considerando la trasformazione isoentropica e la pressione identica alla pressione misurata nell'impianto reale.
Anche l'entropia del vapore spillato sarà quindi pari a quella dello stato 3 e 4 ossia 6.837 kJ/kgK, ricaviamo gli altri valori dello stato.

Per permettere un corretto mescolamento, la pompa di estrazione deve portare l'acqua dallo stato 1 ad uno stato 1' con una pressione di 5.32 atm, il processo si assume
isoentropico.

Considerando la trasformazione adiabatica nell'estrattore di vapore è possibile fare un bilancio sull'entalpia:
\begin{equation*}
    h_{2'} = \frac{\dot m_{1'}\cdot h_{1'} + \dot m_{SP2} \cdot h_{SP2}}{\dot m_{2'}} = 222.5\ kJ/kg
\end{equation*}
Considerata la pressione dello stato 2' identica a quella del vapore spillato e calcolata l'entalpia si possono ricavare le altre variabili di stato.

Lo stato 2 è successivo alla pompa di alimento, trasformazione considerata isoentropica, fissiamo il valore di pressione identico a quello dello stato 3 e calcoliamo
le altre variabili.
\begin{center}
    \begin{tabular}{l|c|c|c|c|c}
        stato    & $p\ (atm)$ & $T\ (\text{°}C) $&$X\ (\%$)& $h\ (\frac{kJ}{kg})$  & $s\ (\frac{kJ}{kg\cdot K})$\\ \hline
        1   &        0.0419 &            30   &    0      & 125.6    &0.4368 \\ \hline     %uscita condensatore 
        1'  &        5.32    &           30.01   &liq. sott.  & 126.3&0.4368 \\ \hline      %uscita pompa estrazione
        2'   &       5.32    &           53.1&liq. sott.& 222.5    &0.743 \\ \hline         %uscita mescolatore
        2   &        59.05   &           53.1&liq. sott.&  228.0   &0.743   \\ \hline       %uscita pompa alimento
        3   &        59.05   &           485  & vap. surr.& 3381     &6.837  \\ \hline  %ingresso turbina
        SP2  &       5.32        &       160.9 &  vap. surr.&2767     &6.837  \\ \hline  %spillamento
        4   &        0.0419       &      29.76&     79.81 & 2064     &6.837                 %uscita turbina
    \end{tabular}
\end{center}

È ora possibile calcolare il rendimento limite definito nella \eqref{eq:redimento_limite}. Per conoscere la potenza limite eseguo un'analisi energetica
ai flussi di massa entranti e uscenti dalla turbina:
\begin{equation*}
    P_T = \dot m_3 \cdot h_3 - \dot m_1 \cdot h_1 - \dot m_{SP2} \cdot h_{SP2} = 25.819\ MW
\end{equation*}
A questa potenza devo sottrarre quella necessaria alle pompe, ricavabile analogamente.
\begin{equation*}
    P_{p_estrazione} = \dot m_1 \cdot \left( h_{1'} - h_1 \right) = 13.4\ kW \\
    P_{p_alimento} = \dot m_2 \cdot \left( h_2 - h_{2'} \right) = 109\ kW 
\end{equation*}
La \textbf{potenza limite} è dunque pari a 25.697 MW.

La potenza termica trasferita al fluido è pari alla portata per il salto di entalpia a monte e a valle del generatore di vapore
\begin{equation*}
    \dot Q_1 = \dot m_2 \cdot (h_3-h_2) = 62.709\ MW
\end{equation*}
Il rendimento limite è dunque pari al 40.98\%

Per il confronto del rendimento limite con il rendimento di Carnot è necessario calcolare le temperature medie di adduzione e sottrazione del calore
utilizzando nuovamente la \eqref{eq:T_ds}.

La temperatura media di adduzione del calore è pari a
\begin{equation*}
    T_{ma} = \frac{h_3 - h_2}{s_3 - s_2} = 517.4\ K = 244.2\ \text{°}C
\end{equation*}
La temperatura media di sottrazione del calore vale invece:
\begin{equation*}
    T_{ms} = \frac{h_4-h_1}{s_4 - s_1} = 302.9\ K = 29.7\ \text{°}C
\end{equation*}
Utilizzando la \eqref{eq:rendimento_carnot} troviamo il rendimento di Carnot pari a 41.46\%.

\subsection{Rendimento interno di impianto}
Ricaviamo ancora una volta la potenza reale erogata dalla turbina, riferendoci questa volta ai dati termodinamici reali.
\begin{equation*}
    P_T = \dot m_3 \cdot h_3 - \dot m_{SP2} \cdot h_{SP2} - \dot m_4 \cdot h_4 = 20.531\ MW
\end{equation*}
Calcolo ora la potenza necessaria ai meccanismi ausiliari, posso trascurare la potenza necessaria alla pompa di estrazione, non conosco in ogni caso lo stato termodinamico
reale successivo alla pompa.
\begin{equation*}
    P_{p_alimento} = \dot m_2 \cdot \left( h_2 - h_{2'} \right) = 250\ kW 
\end{equation*}
La potenza netta reale sarà circa 20.280 MW.

Il rendimento interno vale dunque 78.92\%.

\subsection{Rendimento reale}
È necessario il calcolo della potenza termica trasferita nell'impianto reale per poter valutare il rendimento reale:
\begin{equation*}
    \dot Q_1 = \dot m_2 \cdot (h_{3'} - h_2) = 63.130\ MW
\end{equation*}
Il rendimento reale è dunque 32.12\%.

\subsection{Rendimento meccanico}
Si assume che l'alternatore sia identico all'impianto precedente e così anche il suo rendimento pari a 97.5\%.
La $P_{ua}$ sarà pari quindi a 20.195 MW. Il rendimento meccanico quindi vale 99.58\%.

\subsection{Rendimento adiabatico di espansione}
Valutata nella sezione \ref{subsec:rendimento_limite_2} la potenza ideale disponibile in turbina, è possibile calcolare il rendimento adiabatico di espansione pari a
79.90\%.

\subsection{Verifica dei rendimenti}
\begin{align*}
    \eta_g \stackrel{sch}{=} \eta_b \cdot \eta_l \cdot \eta_{ii} \cdot \eta_m
    = \frac{\dot{Q_1}}{\dot{m}_c H_i}\cdot \frac{P_l}{\dot{Q_1}} \cdot \frac{P_r}{P_l}\cdot \frac{P_{ua}}{P_r} = \\
    =  0.8977 \cdot 0.4098 \cdot 0.7892 \cdot 0.9958 = 28.91 \%
\end{align*}
 %l'impianto 2 viene trattato in un file differente > modifica impianto2.tex e dopo ricompila da qui
%\section{Impianto da 18 MW con 4 spillamenti}
L'ultimo impianto presenta 4 spillamenti per il preriscaldamento del fluido; la potenza netta è necessariamente diminuita
ma un rapido calcolo permette di valutare il rendimento elettrico globale, aumentato ad un valore di 30.01\%.
Il rendimento dell'alternatore si può stimare, anche in questo caso, pari a 97.5\%.
Il rendimento globale vale dunque 30.78\%

\subsection{Rendimento di combustione}
Potenza termica ricavabile dal combustibile: $\dot m_c \cdot H_i = 59.850$ MW.\\
Potenza termica trasferita al fluido: $\dot m_{3'} \cdot (h_{3'} - h_{2R}) = 53.916$ MW.\\
Rendimento di combustione: 90.08\%.

\subsection{Rendimento limite}
Seguiamo ora l'analisi del ciclo limite considerando come valori di riferimento quelli dello stato 1 e 3.
\begin{center}
    \begin{tabular}{r|c|c|c}
        stato    & $p\ (atm)$ & $h\ (kJ/kg)$ & $T\ (\text{°}C) $\\ \hline
        1   &        0.0419 &          125.6   &           30     \\ \hline
        3   &        59.05  &           3381   &           485
    \end{tabular}
\end{center}
Siano SP1, SP2, SP3 ed SP4 gli stati termodinamici consecutivi ai vari spillamenti; si suppone che abbiano tutti la stessa entropia specifica.
Sia fissata ancora una volta la pressione all'interno del mescolatore adiabatico uguale a quella del secondo spillamento, ossia 5.32 atm.
Si può ricavare lo stato 1' consecutivo alla compressione isoentropica provocata dalla pompa di estrazione.

Calcolo quindi gli stati termodinamici degli spillamenti; l'entropia è identica a quella dello stato 3 in ingresso alla turbina.
\begin{center}
    \begin{tabular}{l|c|c|c|c|c}
        stato    & $p\ (atm)$ & $T\ (\text{°}C) $&$X\ (\%$)& $h\ (\frac{kJ}{kg})$  & $s\ (\frac{kJ}{kg\cdot K})$\\ \hline
        SP1   &       8.42 &     211     & vap. surr. &  2863   &6.837 \\ \hline 
        SP1'    &       8.42    &173    &   liq. sott. &732.2   &2.071  \\ \hline
        SP2  &        5.32    &    162    & vap. surr. & 2769  &6.837 \\ \hline    
        SP3  &      1.47       &    111    &  93.30    &2544 &6.837   \\ \hline
        SP4   &    0.55      &      84    &    88.83   & 2393&6.837 
    \end{tabular}
\end{center}

Anche lo stato 4 si può ricavare fissando l'entropia e la pressione successive alla espansione in turbina.

Supponiamo che lo stato 2R sia ad una temperatura pari alla temperatura di saturazione dello spillamento 1 (SP1), pari a 173°C.

Ipotizzando uno scambio termico ideale posso suppore che le temperature dei due fluidi uscenti dal rigeneratore siano identiche, ossia che la temperatura
del fluido spillato dalla turbina, dopo lo scambio termico, abbia la stessa temperatura del fluido in ingresso al generatore di vapore (173°C).

Per valutare la temperatura in uscita dalla pompa (stato 2) posso eseguire un bilancio energetico sul rigeneratore.
\begin{equation*}
    h_2 =\frac{ \dot m_{2R}\cdot h_{2R} - \dot m_{SP2}\cdot(h_{SP2}-h_{SP2'})}{\dot m_{2R}} = 700.6\ kJ/kg
\end{equation*}
Ricavato lo stato 2, si può ricavare lo stato 2' considerando la compressione isoentropica.

\begin{center}
    \begin{tabular}{l|c|c|c|c|c}
        stato    & $p\ (atm)$ & $T\ (\text{°}C) $&$X\ (\%$)& $h\ (\frac{kJ}{kg})$  & $s\ (\frac{kJ}{kg\cdot K})$\\ \hline
        1   &        0.0419 &            30   &    0      & 125.6    &0.4368 \\ \hline     %uscita condensatore 
        1'  &        5.32    &           30.01   &liq. sott.  & 126.3&0.4368 \\ \hline      %uscita pompa estrazione
        2'  &          5.32  &         154&liq. sott.           &694&1.9866  \\ \hline      %ingresso pompa alimento
        2   &          59.05&           165&liq. sott.      &700.6& 1.9866   \\ \hline      %uscita pompa alimento
        2R  &       59.05   &           173     &liq. sott. &  735  & 2.0644 \\ \hline      %ingresso caldaia  
        3   &        59.05  &            485&  vap. surr.    &  3381 &6.837  \\ \hline      %ingresso turbina
        4   &        0.0419       &      29.76&     79.81 &    2064  &6.837                %uscita turbina
    \end{tabular}
\end{center}

Si può calcolare infine la potenza limite erogata dalla turbina pari a 23.569 MW.
\begin{equation*}
    P_T = \dot m_3 h_3 - \dot m_{SP1} h_{SP1} - \dot m_{SP2} h_{SP2} - \dot m_{SP3} h_{SP3} -\dot m_{SP4} h_{SP4} - \dot m_{4} h_4
\end{equation*}

Calcoliamo inoltre la potenza necessaria al funzionamento delle pompe:
\begin{align*}
    P_{p_{estrazione}} = \dot m_1 \cdot (h_{1'} - h_1) = 10.97\ kW \\
    P_{p_{alimento}} = \dot m_2 \cdot (h_2 - h_{2'} ) = 131.27\ kW
\end{align*}

La potenza utile limite sarà dunque 23.427 MW.
La potenza termica trasferita al fluido è:
\begin{equation*}
    \dot Q_1 = \dot m_2 \cdot (h_3 - h_{2R}) = 52.626\ MW
\end{equation*}

Il rendimento limite d'impianto vale dunque 44.52\%.

\subsection{Rendimento di Carnot}
Il calcolo delle temperature medie si esegue mediante la \eqref{eq:T_ds}: 
\begin{align*}
    T_{ma} = \frac{h_3-h_{2R}}{s_3-s_{2R}} = 281.26 \text{°C} \\
    T_{ms} = \frac{h_4-h_1}{s_4-s_1} = 29.71 \text{°C}
\end{align*}

Il rendimento di Carnot è pari dunque a:
\begin{equation*}
    \eta_{Carnot} = 1 - \frac{T_{ms}}{T_{ma}} = 45.37\%
\end{equation*}

\subsection{Rendimento interno di impianto}
Analizzo l'impianto reale valutando le potenze necessarie ed erogate da turbina e pompe tramite gli stati termodinamici reali indicati sullo schema d'impianto:
\begin{align*}
    P_T = \dot m_3 h_3 - (\dot m_{SP1}h_{SP1}+ \dot m_{SP2}h_{SP2}+ \dot m_{SP3}h_{SP3}+ \dot m_{SP4}h_{SP4}+ \dot m_4h_4) \\
    P_{alimento} = \dot m_2R \cdot (h_2 - h_{2'})
\end{align*}
Trascuro il lavoro necessario alle altre due pompe.
La potenza utile reale è pari a 18.747 MW.

Il rendimento interno vale dunque 80.02\%.

\subsection{Rendimento reale}
Calcoliamo la potenza termica reale trasferita nel generatore di vapore:
\begin{equation*}
    \dot Q_1 = \dot m_{2R} \cdot (h_{3'} - h_{2R}) = 53.916\ MW
\end{equation*}
Il rendimento reale è pari a 34.77\%.

\subsection{Rendimento meccanico}
Sia la potenza elettrica generata pari a 18 MW e il rendimento dell'alternatore pari a 97.5\%; la potenza utile all'asse della turbina è pari a 18.46 MW.

Il rendimento meccanico è dunque 98.47\%.

\subsection{Rendimento adiabatico di espansione}
Il rapporto tra le espansioni in turbina reale ed isoentropica, è pari a 80.81\%.

\subsection{Verifica dei rendimenti}
\begin{align*}
    \eta_g \stackrel{sch}{=} \eta_b \cdot \eta_l \cdot \eta_{ii} \cdot \eta_m
    = \frac{\dot{Q_1}}{\dot{m}_c H_i}\cdot \frac{P_l}{\dot{Q_1}} \cdot \frac{P_r}{P_l}\cdot \frac{P_{ua}}{P_r} = \\
    =  0.9008 \cdot 0.4452 \cdot 0.8002 \cdot 0.9847 = 31.60 \%
\end{align*}
 %stesso discorso di prima > impianto3.tex
\end{document}
