\documentclass[a4paper,12pt]{article}
\usepackage[T1]{fontenc}
\usepackage[utf8]{inputenc}
\usepackage{mathtools}
\usepackage[italian]{babel}
%\usepackage{graphicx}
\usepackage{float}
\usepackage{textcomp}
\usepackage{amsmath}

\title{Esercitazione 2}
\author{Olivieri Daniele}
\date{}

\begin{document}
\maketitle
% A partire dai dati sperimentali riportati nei tre schemi d'impianto calcolare
% \begin{itemize}
%     \item I rendimenti:
%     \begin{itemize}
%         \item globale elettrico
%         \item globale
%         \item di combustione
%         \item limite
%         \item interno d'impianto
%         \item reale
%         \item meccanico
%         \item adiabatico di espansione
%     \end{itemize}
%     \item Le temperature medie di adduzione e sottrazione del calore
% \end{itemize}
% Confrontare i risultati.

\section{Definizione dei rendimenti}
A partire dai dati sperimentali riportati nei tre schemi d'impianto calcolare i seguenti rendienti:
\begin{itemize}
    \item Rendimento globale elettrico
    \begin{equation}
        \label{eq:rendimento_elettrico}
        \eta_{g_{el}} \stackrel{def}{=} \frac{P_{el}}{\dot m_c \cdot H_i}
    \end{equation}
    
    \item Rendimento globale
    \begin{equation}
        \label{eq:rendimento_globale}
        \eta_g \stackrel{def}{=} \frac{P_{ua}}{\dot{m}_c \cdot H_i}
   \end{equation}

   \item Rendimento di combustione
   \begin{equation}
       \label{eq:rendimento_combustione}
       \eta_C \stackrel{def}{=} \frac{\dot{Q_1}}{\dot{m}_c \cdot H_i}
   \end{equation}

   \item Rendimento limite
   \begin{equation}
       \label{eq:redimento_limite}
       \eta_l \stackrel{def}{=} \frac{P_l}{\dot{Q_1}}
   \end{equation}

   \item Rendimento interno d'impianto
   \begin{equation}
        \label{eq:rendimento_interno}
       \eta_{ii} \stackrel{def}{=} \frac{P_r}{P_l}
   \end{equation}

   \item Rendimento meccanico
   \begin{equation}
       \label{eq:rendimento_meccanico}
       \eta_m \stackrel{def}{=} \frac{P_{ua}}{P_r}
   \end{equation}

   \item Rendimento reale
   \begin{equation}
       \label{eq:rendimento_reale}
       \eta_r \stackrel{def}{=} \eta_l \cdot \eta_{ii} = \frac{P_l}{\dot{Q_1}} \cdot \frac{P_r}{P_l} = \frac{P_r}{\dot{Q_1}}
   \end{equation}

   \item Rendimento adiabatico di espansione
   \begin{equation}
        \label{eq:rendimento_adiabatico}
        \eta_{ad_e} \stackrel{def}{=} \frac{L_{is}}{L_r} = \frac{1-1/\beta^{\frac{m-1}{m}}}
        {1-1/\beta^{\frac{k-1}{k}}}
   \end{equation}
   oppure utilizzando il rendimento politropico
   \begin{equation}
        \label{eq:rendimento_adiabatico_con_rendimento_politropico}
        \eta_{ad_e} = \frac{1-1/\beta^{\frac{k-1}{k}\eta_{pe}}}
        {1-1/\beta^{\frac{k-1}{k}}}
        %uso \displaystyle per forzare l'integrale grande
   \end{equation}
\end{itemize}
Valutare inoltre le temperature medie di adduzione e sottrazione del calore.

\section{Impianto minimo}
\label{sec:primo_impianto}


\end{document}
