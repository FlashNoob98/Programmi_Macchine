\documentclass[a4paper,12pt]{article}
\usepackage[T1]{fontenc}
\usepackage[utf8]{inputenc}
\usepackage{mathtools}
\usepackage[italian]{babel}
%\usepackage{graphicx}
\usepackage{float}
\usepackage{textcomp}
\usepackage{amsmath}

\title{Esercitazione 1}
\author{Olivieri Daniele}

\begin{document}
\maketitle
Valutare lo scambio di lavoro meccanico e di energia termica
delle seguenti trasformazioni:
\begin{itemize}
    \item Compressione adiabatica isoentropica di 1 kg di aria da
    1 bar e $288.15\ K$ a 2.5 bar.
    
    \item Compressione adiabatica reale di 1 kg di aria da 1 bar
    e 288.15 K a 2.5 bar con $\eta_{pc}$ pari a 0.755
   
    \item Compressione politropica di 1 kg di aria da 1 bar e 288.15 K a 2.5 bar
    con la condizione termodinamica finale coincidente con quella dell'adiabatica reale
   
    \item Compressione isoterma di 1 kg di aria da 1 bar e 288.15 K a 2.5 bar

    \item Compressione di 1 kg di acqua da 1 bar e 288.15 K a 2.5 bar
\end{itemize}

\section{Prima trasformazione}
\label{sec:prima_trasformazione}
Analizziamo la prima trasformazione utilizzando le relazioni per le trasformazioni reversibili,
per prima cosa si determina lo stato del gas prima e dopo l'espansione mediante l'equazione
di stato dei gas
\begin{equation}
    \label{eq:stato_gas}
    PV = RT
\end{equation}
Lo stato iniziale è interamente determinato dato che conosciamo sia la temperatura che la pressione
mentre per il secondo dobbiamo utilizzare la politropica per trasformazioni reversibili,
in questo caso $x$ è proprio uguale a $k$, la costante del gas pari a $Cp/Cv$
\begin{equation}
    \label{eq:politropica}
    p\cdot v^x=\text{cost}
\end{equation}
Possiamo quindi ricavare $V_2$ tramite $$  V_2 = V_1/(\beta^{1/k}) $$

Determinato $V_2$ utilizzando ancora la \eqref{eq:stato_gas} calcoliamo il valore della temperatura
$T_2$ in uscita dal compressore.

Il lavoro necessario alla compressione sarà interamente speso per l'aumento di entalpia del gas
e potrà quindi essere calcolato con
\begin{equation}
    L_{is} = m\cdot \Delta h = m\cdot C_p (T_2 - T_1)
\end{equation}
esso sarà pari a $86.65\ kJ$

Considerando la trasformazione adiabatica, il calore scambiato sarà nullo.

Tabella degli stati

\begin{center}
\begin{tabular}{r|c|c|c}
    stato    & $P\ (bar)$ & $V\ (m^3/kg)$ & $T\ (\text{°}C) $\\ \hline
    1   &           1 &          0.827    &           15     \\ \hline
    2   &         2.5 &          0.429    &           101.2
\end{tabular}
\end{center}

\section{Seconda trasformazione}
\label{sec:seconda_trasformazione}
Anche in questo caso la trasformazione è adiabatica ma viene fornito un valore del rendimento
politropico di compressione $\eta_{pc} = 0.755$, definito come
\begin{equation}
    \eta_{pc} \stackrel{def}{=} \frac{\displaystyle\frac{n}{n-1} R T_1 \left(1-\beta^{\displaystyle\frac{n-1}{n}} \right)}{C_p \left(T_1 - T_2\right)}
    = \frac{L_{pc}}{L_r}
\end{equation}
o equivalentemente 
\begin{equation}
    \eta_{pc} = \frac{n}{n-1} \frac{k-1}{k}
\end{equation}
si può quindi ricavare il valore dell'esponente n della politropica oppure sostituire direttamente
il rendimento politropico nella definizione del rendimento adiabatico e quindi calcolarne il valore.
\begin{equation}
    \label{eq:rendimento_compressione_adiabatico}
    \eta_{ad-c} \stackrel{def}{=} \frac{L_{is}}{L_r} = \frac{ C_p T_1 \left(1-\beta^{\displaystyle\frac{k-1}{k}}\right)}
    {C_p T_1 \left(1-\beta^{\displaystyle\frac{n-1}{n}}\right)} = \frac{1-\beta^{\displaystyle\frac{k-1}{k}}}{1-\beta^{\displaystyle\frac{k-1}{k\eta_{pc}}}}
\end{equation}
svolgendo i calcoli si trova quindi un valore del rendimento adiabatico pari a $\eta_{ad-c}=0.722$.
Il lavoro necessario alla trasformazione adiabatica reale sarà quindi il rapporto tra il lavoro necessario
alla precedente trasformazione isoentropica e il rendimento adiabatico
\begin{equation*}
    L_r = \frac{L_{is}}{\eta_{ad-c}}
\end{equation*}
e sarà pari a $120\ kJ$

\end{document}