\documentclass[a4paper,12pt]{article}
\usepackage[T1]{fontenc}
\usepackage[utf8]{inputenc}
\usepackage{mathtools}
\usepackage[italian]{babel}
%\usepackage{graphicx}
\usepackage{float}
\usepackage{textcomp}
\usepackage{amsmath}

\title{Esercitazione 1}
\author{Olivieri Daniele}
\date{}

\begin{document}
\maketitle
Valutare lo scambio di lavoro meccanico e di energia termica
delle seguenti trasformazioni:
\begin{itemize}
    \item Compressione adiabatica isoentropica di 1 kg di aria da
    1 bar e $288.15\ K$ a 2.5 bar.
    
    \item Compressione adiabatica reale di 1 kg di aria da 1 bar
    e 288.15 K a 2.5 bar con $\eta_{pc}$ pari a 0.755
   
    \item Compressione politropica di 1 kg di aria da 1 bar e 288.15 K a 2.5 bar
    con la condizione termodinamica finale coincidente con quella dell'adiabatica reale
   
    \item Compressione isoterma di 1 kg di aria da 1 bar e 288.15 K a 2.5 bar

    \item Compressione di 1 kg di acqua da 1 bar e 288.15 K a 2.5 bar
\end{itemize}

\section{Trasformazione isoentropica}
\label{sec:prima_trasformazione}
Analizziamo la prima trasformazione utilizzando le relazioni per le trasformazioni reversibili,
per prima cosa si determina lo stato del gas prima e dopo l'espansione mediante l'equazione
di stato dei gas
\begin{equation}
    \label{eq:stato_gas}
    pv = RT
\end{equation}
Lo stato iniziale è interamente determinato dato che conosciamo sia la temperatura che la pressione
mentre per il secondo dobbiamo utilizzare la politropica per trasformazioni reversibili,
in questo caso $x$ è proprio uguale a $k$, la costante del gas pari a $Cp/Cv$
\begin{equation}
    \label{eq:politropica}
    p\cdot v^x=\text{cost}
\end{equation}
Possiamo quindi ricavare $v_2$ tramite $$  v_2 = v_1/(\beta^{1/k}) $$

Determinato $v_2$ utilizzando ancora la \eqref{eq:stato_gas} calcoliamo il valore della temperatura
$T_2$ in uscita dal compressore.

Il lavoro necessario alla compressione sarà interamente speso per l'aumento di entalpia del gas
e potrà quindi essere calcolato con
\begin{equation}
    L_{is} = m\cdot \Delta h = m\cdot C_p (T_2 - T_1)
\end{equation}
esso sarà pari a $86.65\ kJ$

Considerando la trasformazione adiabatica, il calore scambiato sarà nullo.

Tabella degli stati

\begin{center}
    \begin{tabular}{r|c|c|c}
        stato    & $p\ (bar)$ & $v\ (m^3/kg)$ & $T\ (\text{°}C) $\\ \hline
        1   &           1 &          0.827    &           15     \\ \hline
        2   &         2.5 &          0.429    &           101.2
    \end{tabular}
\end{center}

\section{Trasformazione adiabatica reale}
\label{sec:seconda_trasformazione}
Anche in questo caso la trasformazione è adiabatica ma viene fornito un valore del rendimento
politropico di compressione $\eta_{pc} = 0.755$, definito come
\begin{equation}
    \label{eq:rendimento_politropica}
    \eta_{pc} \stackrel{def}{=} \frac{\displaystyle\frac{n}{n-1} R T_1 \left(1-\beta^{\displaystyle\frac{n-1}{n}} \right)}{C_p \left(T_1 - T_2\right)}
    = \frac{L_{pc}}{L_r}
\end{equation}
o equivalentemente 
\begin{equation}
    \label{eq:rendimento_politropica_breve}
    \eta_{pc} = \frac{n}{n-1} \frac{k-1}{k}
\end{equation}
si può quindi ricavare il valore dell'esponente n della politropica oppure sostituire direttamente
il rendimento politropico nella definizione del rendimento adiabatico e quindi calcolarne il valore.
\begin{equation}
    \label{eq:rendimento_compressione_adiabatico}
    \eta_{ad-c} \stackrel{def}{=} \frac{L_{is}}{L_r} = \frac{ C_p T_1 \left(1-\beta^{\displaystyle\frac{k-1}{k}}\right)}
    {C_p T_1 \left(1-\beta^{\displaystyle\frac{n-1}{n}}\right)} = \frac{1-\beta^{\displaystyle\frac{k-1}{k}}}{1-\beta^{\displaystyle\frac{k-1}{k\eta_{pc}}}}
\end{equation}
svolgendo i calcoli si trova quindi un valore del rendimento adiabatico pari a $\eta_{ad-c}=0.722$.
Il lavoro necessario alla trasformazione adiabatica reale sarà quindi il rapporto tra il lavoro necessario
alla precedente trasformazione isoentropica e il rendimento adiabatico
\begin{equation*}
    L_r = \frac{L_{is}}{\eta_{ad-c}}
\end{equation*}
e sarà pari a $120\ kJ$.
Anche in questo caso il calore scambiato è considerato nullo.

\section{Trasformazione politropica}
\label{sec:terza_trasformazione}
La terza trasformazione richiede il calcolo delle condizioni termodinamiche dello stato finale della
compressione adiabatica reale, possiamo calcolare la politropica passante per gli stessi punti dato che
ci viene fornito il rendimento.
Riferendoci quindi alla \eqref{eq:politropica} dobbiamo calcolare il valore dell'esponente incognito
ricavabile dalla \eqref{eq:rendimento_politropica_breve} che sarà uguale a
\begin{equation}
    n = \frac{\eta_{pc}}{\eta_{pc}-\displaystyle\frac{k-1}{k}}
\end{equation}
in questo caso pari a 1.609, maggiore del valore $k = 1.4$ per l'aria, com'era da aspettarsi.
Rieseguendo i calcoli svolti nella sezione \ref{sec:prima_trasformazione} possiamo creare la nuova tabella
degli stati termodinamici:

\begin{center}
    \begin{tabular}{r|c|c|c}
        stato    & $p\ (bar)$ & $v\ (m^3/kg)$ & $T\ (\text{°}C) $\\ \hline
        1   &           1 &          0.827    &           15     \\ \hline
        2   &         2.5 &          0.468    &           134.4
    \end{tabular}
\end{center}
temperatura e volume specifico sono maggiori rispetto alle condizioni successive alla trasforamzione
isoentropica.
Il lavoro necessario per la trasformazione politropica è ricavabile dalla definizione del rendimento politropico
\eqref{eq:rendimento_politropica} ed è pari a $90.6\ kJ$, per raggiungere lo stato termodinamico 2 però
è necessario fornire una quantità di calore al gas pari alla differenza tra il lavoro reale e quello politropico
ossia $29.4\ kJ$ di calore.

\section{Trasformazione isoterma}
\label{sec:quarta_trasformazione}
La compressione isoterma implica una sottrazione di calore continua al fine di mantenere la temperatura
costante durante la compressione, tecnicamente irrealzzabile a causa della geometria dei compressori
fortemente adiabatici, si può ottenere invece una interrefrigerazione dividendo la compressione in più 
stadi.
Utilizzando ancora la \eqref{eq:politropica} e ponendo l'esponente pari ad 1 si ottiene l'equazione dell'isoterma
\begin{equation}
    p\cdot v = \text{cost}
\end{equation}
Ricaviamo quindi gli stati termodinamici come fatto in precedenza
\begin{center}
    \begin{tabular}{r|c|c|c}
        stato    & $p\ (bar)$ & $v\ (m^3/kg)$ & $T\ (\text{°}C) $\\ \hline
        1   &           1 &          0.827    &           15     \\ \hline
        2   &         2.5 &          0.331    &           15
    \end{tabular}
\end{center}
Il lavoro necessario alla compressione è pari al calore scambiato dal sistema dato che l'energia interna $U$
di un gas perfetto è funzione della sola temperatura, resta quindi anch'essa costante, ciò implica che
$Q = L$.
Il lavoro è facilmente calcolabile come 
\begin{equation}
    L = \int_{v_1}^{v_2} p\cdot dV
\end{equation}
utilizzando la \eqref{eq:stato_gas} e sostituendo $p$ si ricava:
\begin{equation}
    L = mRT \ln\left(\frac{1}{\beta}\right)
\end{equation}
Il lavoro di compressione isotermo è quindi pari a $75.8\ kJ$ così come anche il calore uscente necessario a 
mantenere la temperatura costante.

\section{Compressione di un liquido}
\label{sec:compressione_liquido}
Valutare lo scambio di lavoro meccanico ed energia termica necessari alla compressione di 1 kg di acqua da
1 bar e $288.15 K$ a 2.5 bar.

\end{document}
