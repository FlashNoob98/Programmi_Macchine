\documentclass[a4paper,10pt]{article}
\usepackage[T1]{fontenc}
\usepackage[utf8]{inputenc}
\usepackage{mathtools}
\usepackage[italian]{babel}
\usepackage{graphicx}
\usepackage{textcomp}

\title{Appunti personali}
\author{Olivieri Daniele}
\begin{document}
\maketitle
\section{Potere calorifico}
\label{sec:potere_calorifico}
%\subsection{Definizione}
%\label{sec:potere_calorifico_def}
Esistono due definizioni di potere calorifico: superiore ed inferiore.

Il potere calorifico \textbf{superiore}, indicato con $\Delta_c H_s\text{°}$, è la 
quantità totale di calore estraibile da un combustibile, esso tiene conto del calore 
latente necessario alla vaporizzazione dell'acqua formatasi con la cobustione.

Il potere calorifico \textbf{inferiore}, indicato con $\Delta_c H_i\text{°}$, invece 
non tiene conto del calore necessario alla vaporizzazione dell'acqua, indica quindi 
la quantità massima di energia convertibile idealmente dal combustibile, viene 
utilizzato infatti nella definizione del rendimento globale.

Valori tipici del potere calorifico per il metano sono: $\Delta_c H_s\text{°}\ =\ 
55.50\ MJ/kg$, $\Delta_c H_i\text{°}\ =\ 50.00\ MJ/kg$

\section{Combustione}
\label{sec:combustione}
Il processo di combustione degli idrocarburi in aria $\left(78\%\ N_2\ -\ 21\%\ O_2 
\right)$ avviene seguendo la seguente relazione stechiometrica
\begin{equation}
C_n H_m\ +\ \left(n+\frac{m}{4}\right)\left(O_2+3.773N_2\right)\ \rightarrow\ nCO_2\ 
+\ \frac{m}{2}H_2O
\end{equation}
Come si può vedere c'è produzione di acqua che evaporando sottrae energia termica.

\section{Rendimento}
\label{sec:rendimento}
Il \textbf{rendimento globale} di un impianto motore è definito dalla seguente
\begin{equation}
 \eta_g \stackrel{def}{=} \frac{P_{ua}}{\dot{m}_c H_i}
\end{equation}
dove $P_{ua}$ è la \textbf{potenza utile all'asse} $\left(kW\right)$, $\dot{m}_c$ è 
la \textbf{portata massica} di combustibile $\left(\frac{kg}{s}\right)$ e $H_i$ è il 
\textbf{potere calorifico inferiore} prima definito $\left(\frac{kJ}{kg}\right)$.
L'inverso del rendimento è definito come \textbf{consumo specifico di calore}
\begin{equation}
C_{s_q} \stackrel{def}{=} \frac{\dot{m}_c H_i}{P_{ua}} = \frac{m_c H_i}{L_{ua}}
\end{equation}

Utilizzando la \textbf{potenza elettrica} generata anzichè quella meccanica
(spesso confondibili dato l'elevato rendimento degli alternatori) si definisce
l'\textbf{heat rate}
\begin{equation}
 C_{s_{el}} \stackrel{def}{=} \frac{\dot{m}_c H_i}{P_{el}} = \frac{m_c H_i}{E_{el}} \ 
 \ \ \left(\frac{kJ}{kWh} \right)
\end{equation}

Fissato il tipo di combustibile si definisce il \textbf{consumo specifico di 
combustibile}
\begin{equation}
C_{s_c} \stackrel{def}{=} \frac{\dot{m}_c}{P_{ua}} = \frac{m_c}{L_{ua}} = \frac{1}
{\eta_gH_i} \ \ \ \left(\frac{g}{kWh} \right)
\end{equation}
%rivedi le analisi dimensionali
\subsection*{Schematizzazione del rendimento}
\label{sec:rendimento_schema}
Essendo il rendimento globale un parametro così generale per un impianto e povero
di informazioni dettagliate, è comodo schematizzarlo e suddividerlo in 4 parti
\begin{equation}
\label{eq:rendimento_schema_eq}
 \eta_g \stackrel{sch}{=} \frac{\dot{Q_1}}{\dot{m}_c H_i}\cdot \frac{P_l}{\dot{Q_1}}
 \cdot \frac{P_r}{P_l}\cdot \frac{P_{ua}}{P_r}
\end{equation}

Analizziamo i 4 rapporti presenti nella \eqref{eq:rendimento_schema_eq}
\begin{enumerate}
\item \textbf{Rendimento di combustione}: potenza termica trasferita al fluido 
rispetto a quella disponibile nel combustibile

\item \textbf{Rendimento limite}:
%potenza ricavata se il fluido eseguisse un 
%ciclo termodinamico reversibile rispetto alla potenza termica trasferita al
%fluido
potenza ricavabile se il fluido reale eseguisse un ciclo limite, rispetto alla 
potenza termica ricevuta

\item \textbf{Rendimento interno} (o specifico): potenza meccanica ottenuta 
dall'impianto reale rispetto a quello limite

\item \textbf{Rendimento meccanico}: rapporto tra la potenza meccanica effettivamente 
disponibile rispetto a quella complessivamente prodotta, alla quale va sottratta la 
potenza necessaria al funzionamento dei dispositivi ausiliari e le perdite per 
attrito
\end{enumerate}

Moltiplicando il secondo e terzo termine della \eqref{eq:rendimento_schema_eq} si 
definisce il \textbf{rendimento termico reale}
\begin{equation}
\label{eq:rendimento_reale}
 \frac{P_l}{\dot{Q_1}}\cdot \frac{P_r}{P_l} = \frac{P_r}{\dot{Q_1}} \stackrel{def}{=} 
 \eta_r
\end{equation}


\end{document}
