\documentclass[a4paper,10pt]{article}
\usepackage[T1]{fontenc}
\usepackage[utf8]{inputenc}
\usepackage{mathtools}
\usepackage[italian]{babel}
\usepackage{graphicx}
\usepackage{textcomp}

\textwidth = 424pt
\title{Appunti personali}
\author{Olivieri Daniele}
\begin{document}
\maketitle
\section{Potere calorifico}
\label{sec:potere_calorifico}
%\subsection{Definizione}
%\label{sec:potere_calorifico_def}
Esistono due definizioni di potere calorifico: superiore ed inferiore.

Il potere calorifico \textbf{superiore}, indicato con $\Delta_c H_s\text{°}$, è la quantità totale di calore estraibile da un combustibile, esso tiene conto del calore latente necessario alla vaporizzazione dell'acqua formatasi con la cobustione.

Il potere calorifico \textbf{inferiore},indicato con $\Delta_c H_i\text{°}$, invece non tiene conto del calore necessario alla vaporizzazione dell'acqua, indica quindi la quantità massima di energia convertibile idealmente dal combustibile, viene utilizzato infatti nella definizione del rendimento globale.

Valori tipici del potere calorifico per il metano sono: $\Delta_c H_s\text{°}\ =\ 55.50\ MJ/kg$, $\Delta_c H_i\text{°}\ =\ 50.00\ MJ/kg$

\section{Combustione}
\label{sec:combustione}
Il processo di combustione degli idrocarburi in aria $\left(78\%\ N_2\ -\ 21\%\ O_2 \right)$ avviene seguendo la seguente relazione stechiometrica
\[C_n H_m\ +\ \left(n+\frac{m}{4}\right)\left(O_2+3.773N_2\right)\ \rightarrow nCO_2\ +\ \frac{m}{2}H_2O \]
Come si può vedere c'è produzione di acqua che evaporando sottre energia termica.

\end{document}
