\documentclass[a4paper,12pt]{article}
\usepackage[T1]{fontenc}
\usepackage[utf8]{inputenc}
\usepackage{mathtools}
\usepackage[italian]{babel}
%\usepackage{graphicx}
\usepackage{float}
\usepackage{textcomp}
\usepackage{amsmath}

\title{Esercitazione 5}
\author{Olivieri Daniele}
\date{}

\begin{document}
\maketitle
\begin{center}
    SI Engine 1242 \\
    \begin{tabular}{|l | l|}
        \hline
        Engine model    &Four-Stroke SI 16 valve \\ \hline
        Displacement, Comp.Ratio &  1242 $cm^3$, 10.2:1 \\ \hline
        Bore, Stroke, Con. Rod & 70.80, 78.86, 129 mm \\ \hline
        Intake Valve Open/Close & 20° BTDC / 52°ABDC \\ \hline
        Exh. Valve Open/Close & 53° BBDC / 19° ATDC \\ \hline
        Int./ Exh. max. Valve Lift & 7.5 mm / 7.5 mm \\ \hline
        Maximum Power & 54 kW @5500 rpm \\ \hline
        Maximum Torque & 109 Nm @ 4000 rpm \\ \hline
        Brake Specific Fuel Cons. & 277 g/kWh @Max. Torque \\ \hline        
    \end{tabular}
\end{center}
Considerando le caratteristiche del motore in tabella, stimare o calcolare:
\begin{itemize}
    \item Il rendimento globale massimo $\eta_g$
    \item La Coppia a 5500 giri/min
    \item La potenza a massima Coppia (a 4000 giri/min)
    \item La pmi e la pme a massima Coppia e a massima Potenza
    \item Il rendimento globale a massima Potenza (5500 giri/min)
    \item La velocità media del pistone a massima Coppia e a massima Potenza
\end{itemize}
\section{Rendimento globale}
\label{sec:rendimento_globale}
Stimo per prima cosa il \textbf{rendimento globale massimo} definito come:
\begin{equation}
    \label{eq:rendimento_globale}
    \eta_g \stackrel{def}{=} \frac{P_{eff}}{\dot m_c \cdot H_i}
\end{equation}
In tabella ci viene fornito il valore di consumo specifico di 277 grammi di combustibile per kWh alla massima coppia.
Ipotizzando di utilizzare la benzina essa ha un \textbf{potere calorifico} inferiore di 43.6 MJ/kg, con i seguenti passaggi si ricava il rendimento:
\begin{equation*}
    \eta_g = \frac{1000\cdot3600}{C_s \cdot H_i}
\end{equation*}
il risultato è un rendimento del 29.8\%

\section{Coppia a massima Potenza}
\label{sec:coppia_a_massima_potenza}
La potenza all'asse si misura come il prodotto della coppia per la velocità angolare
\begin{equation}
    \label{eq:potenza_da_coppia}
    P = Nm \cdot \omega
\end{equation}
di conseguenza la coppia sarà il rapporto tra la potenza e la velocità angolare espressa in rad/s, per convertire la velocità angolare bisogna prima moltiplicarla
per $\pi/30$.

La coppia sarà data dunque dalla seguete:
\begin{equation}
    Nm = \frac{P\cdot 30 }{\omega\cdot \pi}
\end{equation}
e pari a 93.76 Nm.

\section{Potenza a massima Coppia}
Utilizzando nuovamente la \eqref{eq:potenza_da_coppia} si può ricavare la potenza del motore quando la coppia è massima (cioè a 4000 giri/min).
\begin{equation*}
    P = \frac{109\cdot4000\cdot\pi}{30} = 45.66\ \text{kW}
\end{equation*}

\section{Pressione media indicata e pressione media effettiva}
La pressione media indicata è definita come il rapporto fra il lavoro indicato, ossia l'integrale del lavoro ricavato dal diagramma indicato $P-\theta$ e la cilindrata
del motore, ossia il volume d'aria teoricamente aspirabile in un ciclo.
\begin{equation}
    p_{mi} \stackrel{def}{=} \frac{L_i}{V}
\end{equation}
Il lavoro indicato è dato dal rapporto tra la potenza e il numero di cicli termodinamici svolti in un secondo, per un motore a 4 tempi il ciclo termodinamico si conclude 
ogni 2 giri del pistone, devo dividere per questo numero $\varepsilon$ la velocità in radianti.
\begin{equation}
    P_i = L_i \frac{n}{60\varepsilon} = p_{mi} \cdot V \cdot \frac{n}{60\varepsilon}
\end{equation}
La pressione media indicata si calcola dunque usando
\begin{equation}
    \label{eq:pressione_media_indicata}
    p_{mi} = \frac{P_i}{V} \cdot  \frac{60\varepsilon}{n}
\end{equation}
La pressione media effettiva invece tiene conto delle perdite meccaniche ed è minore di quella indicata, si ricava moltiplicando la pressione media indicata
per il rendimento meccanico del motore.
\begin{equation}
    P_{me} \stackrel{def}{=} P_{mi}\cdot \eta_m
\end{equation}
Ipotizziamo un rendimmento meccanico di 0.85 in condizioni di massima coppia e di 0.80 in condizioni di massima potenza.
\subsection{Massima Coppia}
La \eqref{eq:pressione_media_indicata} vale 11.03 bar quando il motore si trova in condizioni di massima coppia, ossia la potenza è pari a 45.66 kW a 4000 rpm.
La pressione media effettiva assume il valore di 9.37 bar.
\subsection{Massima Potenza}
La \eqref{eq:pressione_media_indicata} vale 9.49 bar quando il motore si trova in condizioni di massima potenza a 5500 rpm.
La pressione media effettiva vale 7.59 bar.


\end{document}
