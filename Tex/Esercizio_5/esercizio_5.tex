\documentclass[a4paper,12pt]{article}
\usepackage[T1]{fontenc}
\usepackage[utf8]{inputenc}
\usepackage{mathtools}
\usepackage[italian]{babel}
%\usepackage{graphicx}
\usepackage{float}
\usepackage{textcomp}
\usepackage{amsmath}

\title{Esercitazione 5}
\author{Olivieri Daniele}
\date{}

\begin{document}
\maketitle
\begin{center}
    SI Engine 1242 \\
    \begin{tabular}{|l | l|}
        \hline
        Engine model    &Four-Stroke SI 16 valve \\ \hline
        Displacement, Comp.Ratio &  1242 $cm^3$, 10.2:1 \\ \hline
        Bore, Stroke, Con. Rod & 70.80, 78.86, 129 mm \\ \hline
        Intake Valve Open/Close & 20° BTDC / 52°ABDC \\ \hline
        Exh. Valve Open/Close & 53° BBDC / 19° ATDC \\ \hline
        Int./ Exh. max. Valve Lift & 7.5 mm / 7.5 mm \\ \hline
        Maximum Power & 54 kW @5500 rpm \\ \hline
        Maximum Torque & 109 Nm @ 4000 rpm \\ \hline
        Brake Specific Fuel Cons. & 277 g/kWh @Max. Torque \\ \hline        
    \end{tabular}
\end{center}
Considerando le caratteristiche del motore in tabella, stimare o calcolare:
\begin{itemize}
    \item Il rendimento globale massimo $\eta_g$
    \item La Coppia a 5500 giri/min
    \item La potenza a massima Coppia (a 4000 giri/min)
    \item La pmi e la pme a massima Coppia e a massima Potenza
    \item Il rendimento globale a massima Potenza (5500 giri/min)
    \item La velocità media del pistone a massima Coppia e a massima Potenza
\end{itemize}
\section{Rendimento globale}
\label{sec:rendimento_globale}
Stimo per prima cosa il \textbf{rendimento globale massimo} definito come:
\begin{equation}
    \label{eq:rendimento_globale}
    \eta_g \stackrel{def}{=} \frac{P_{eff}}{\dot m_c \cdot H_i}
\end{equation}
In tabella ci viene fornito il valore di consumo specifico di 277 grammi di combustibile per kWh alla massima coppia.
Ipotizzando di utilizzare la benzina essa ha un \textbf{potere calorifico} inferiore di 43.6 MJ/kg, con i seguenti passaggi si ricava il rendimento:
\begin{equation*}
    \eta_g = \frac{1000\cdot3600}{C_s \cdot H_i}
\end{equation*}
il risultato è un rendimento del 29.8\%

\section{Coppia a massima Potenza}
\label{sec:coppia_a_massima_potenza}
\end{document}
