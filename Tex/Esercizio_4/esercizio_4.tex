\documentclass[a4paper,12pt]{article}
\usepackage[T1]{fontenc}
\usepackage[utf8]{inputenc}
\usepackage{mathtools}
\usepackage[italian]{babel}
%\usepackage{graphicx}
\usepackage{float}
\usepackage{textcomp}
\usepackage{amsmath}

\title{Esercitazione 4}
\author{Olivieri Daniele}
\date{}

\begin{document}
\maketitle
In un serbatoio è contenuta dell'aria ad una pressione $p_a$ di 2.58 bar ed una temperatura $T_a$ di 378 K. La pressione ambiente $p_2$ è di 1 bar.
\begin{enumerate}
    \item Valutare le trasformazioni che avvengono in uno stadio di turbina assiale nel quale espande il suddetto fluido, nell'ipotesi
    di trasformazioni adiabatiche isoentropiche e flusso monodimensionale.
    In particolare si valuti lo stato termodinaico statico e totale monte/valle statore e monte/valle rotore. Si consideri sia uno stadio ad azione
    ed uno a reazione (triangoli delle velocità "simmetrici")    
    \item Calcolare il salto entalpico disponibile, $\Delta h^*$   
    \item Calcolare il lavoro ricavato dalla turbina
    \item Calcolare il rendimento di palettatura
    \item Valutare e disegnare i triangoli delle velocità
    \item Rappresentare sul piano T-s ed h-s le trasformazioni
    \item Disegnare le palettature statoriche e rotoriche
\end{enumerate}
Si assuma un angolo tra la velocità assoluta all'uscita dello statore e la direzione tangenziale $\alpha_1$ di 15°
\section{Analisi condizioni termodinamiche}
\label{sec:analisi_termodinamiche}
Il gas si trova inizialmente alle condizioni di ristagno (A), supposta nulla la velocità all'interno del serbatoio, espande isoentropicamente fino alla pressione di 1 bar
attraverso uno stadio di turbina che ne ricaverà dunque del lavoro da tale espansione. Non conoscendo le dimensioni del serbatoio o la massa di gas possiamo fare
solo ragionamenti per unità di massa.
Considerando la trasformazione isoentropica possiamo facilmente calcolare lo stato termodinamico 2 in uscita dalla turbina.
\begin{equation}
    T_2 = T_a\cdot\beta^{\frac{1-k}{k}}
\end{equation}
con $\beta$ pari a 2.58 e $k$ pari a 1.4, la temperatura in uscita dalla turbina varrà 288.3 K, con l'ausilio dell'equazione di stato dei gas
possiamo quindi completare la tabella con gli stati termodiamici a monte e a valle della turbina.

\begin{center}
    \begin{tabular}{c|c|c|c}
        Stato   &p (bar)    &T (°C) &v ($m^3/kg$) \\ \hline
        A       &2.58       &104,8  &0.420  \\
        2       &1          &15.2   &0.827  
    \end{tabular}
\end{center}
Il salto entalpico specifico disponibile $\Delta h^*$ è quindi pari a $C_p \cdot (T_1-T_2)$ ossia 90.12 kJ/kg

\section{Stadio ad azione}
\label{sec:stadio_ad_azione}
Consideriamo in prima analisi uno stadio ad azione, l'espansione del gas avviene nelle palette statoriche che trasformano tutta l'energia di pressione in velocità,
essendo ferme non producono ovviamente lavoro, si individua quindi una condizione termodinamica intermedia tra statore e rotore nella quale la pressione è identica
alla pressione in uscita ma l'energia cinetica ha eguagliato l'energia potenziale (ossia l'entalpia totale $\Delta h^*$) posseduta in precedenza dal gas.
\begin{equation}
    \label{eq:entalpia_stadio_azione}
    \Delta h^* = \frac{c_1^2}{2}
\end{equation}
quindi
\begin{equation}
    \label{eq:vel_stadio_azione}
    c_1 = \sqrt{2\Delta h^*}    
\end{equation}
$c_1$ = 13.43 m/s

Il gas deve essere quindi rallentato fino alla velocità $c_2$ dalle palette rotoriche per poter ottenere un lavoro utile, per massimizzare tale lavoro la velocità
in uscita dalla turbina deve essere la minima possibile ma al limite diversa da 0 per conservare la condizione di flusso stazionario (non può esserci accumulo).
Il lavoro ottenuto dall'espansione sarà quindi
\begin{equation}
    l = \Delta h^* - \frac{c_2^2}{2}
\end{equation}
Sia $\vec u$ la velocità tangenziale del rotore, $\vec c$ la velocità assoluta del fluido rispetto ad un sistema inerziale esterno alla turbina,
risulterà $\vec w$ la velocità relativa del fluido rispetto alla palettatura e pari quindi a  $\vec c - \vec u$
\end{document}
