\documentclass[a4paper,12pt]{article}
\usepackage[T1]{fontenc}
\usepackage[utf8]{inputenc}
\usepackage{mathtools}
\usepackage[italian]{babel}
%\usepackage{graphicx}
\usepackage{float}
\usepackage{textcomp}
\usepackage{amsmath}

\title{Esercitazione 4}
\author{Olivieri Daniele}
\date{}

\begin{document}
\maketitle
In un serbatoio è contenuta dell'aria ad una pressione $p_01$ di 2.58 bar ed una temperatura $T_01$ di 378 K. La pressione ambiente è di 1 bar.
\begin{enumerate}
    \item Valutare le trasformazioni che avvengono in uno stadio di turbina assiale nel quale espande il suddetto fluido, nell'ipotesi
    di trasformazioni adiabatiche isoentropiche e flusso monodimensionale.
    In particolare si valuti lo stato termodinaico statico e totale monte/valle statore e monte/valle rotore. Si consideri sia uno stadio ad azione
    ed uno a reazione (triangoli delle velocità "simmetrici")    
    \item Calcolare il salto entalpico disponibile, $\Delta h^*$   
    \item Calcolare il lavoro ricavato dalla turbina
    \item Calcolare il rendimento di palettatura
    \item Valutare e disegnare i triangoli delle velocità
    \item Rappresentare sul piano T-s ed h-s le trasformazioni
    \item Disegnare le palettature statoriche e rotoriche
\end{enumerate}
Si assuma un angolo tra la velocità assoluta all'uscita dello statore e la direzione tangenziale $\alpha_1$ di 15°


\end{document}
