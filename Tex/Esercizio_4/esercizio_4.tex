\documentclass[a4paper,12pt]{article}
\usepackage[T1]{fontenc}
\usepackage[utf8]{inputenc}
\usepackage{mathtools}
\usepackage[italian]{babel}
%\usepackage{graphicx}
\usepackage{float}
\usepackage{textcomp}
\usepackage{amsmath}

\title{Esercitazione 4}
\author{Olivieri Daniele}
\date{}

\begin{document}
\maketitle
In un serbatoio è contenuta dell'aria ad una pressione $p_{01}$ di 2.58 bar ed una temperatura $T_{01}$ di 378 K. La pressione ambiente $p_{02}$ è di 1 bar.
\begin{enumerate}
    \item Valutare le trasformazioni che avvengono in uno stadio di turbina assiale nel quale espande il suddetto fluido, nell'ipotesi
    di trasformazioni adiabatiche isoentropiche e flusso monodimensionale.
    In particolare si valuti lo stato termodinaico statico e totale monte/valle statore e monte/valle rotore. Si consideri sia uno stadio ad azione
    ed uno a reazione (triangoli delle velocità "simmetrici")    
    \item Calcolare il salto entalpico disponibile, $\Delta h^*$   
    \item Calcolare il lavoro ricavato dalla turbina
    \item Calcolare il rendimento di palettatura
    \item Valutare e disegnare i triangoli delle velocità
    \item Rappresentare sul piano T-s ed h-s le trasformazioni
    \item Disegnare le palettature statoriche e rotoriche
\end{enumerate}
Si assuma un angolo tra la velocità assoluta all'uscita dello statore e la direzione tangenziale $\alpha_1$ di 15°

Il gas si trova inizialmente alle condizioni di ristagno, supposta nulla la velocità all'interno del serbatoio, espande isoentropicamente fino alla pressione di 1 bar
attraverso uno stadio di turbina che ne ricaverà dunque del lavoro da tale espansione. Non conoscendo le dimensioni del serbatoio o la massa di gas possiamo fare solo ragionamenti
per unità di massa.
COnsiderando la trasformazione isoentropica possiamo facilmente calcolare lo stato termodinamico 2 in uscita dalla turbina.
\begin{equation}
    T_2 = T_{01}\cdot\beta^{\frac{1-k}{k}}
\end{equation}
con $\beta$ pari a 2.58 e $k$ pari a 1.4, la temperatura in uscita dalla turbina varrà 288.3 K, con l'ausilio dell'equazione di stato dei gas
possiamo quindi completare la tabella con gli stati termodiamici a monte e a valle della turbina.

\begin{center}
    \begin{tabular}{c|c|c|c}
        Stato   &p (bar)    &T (°C) &v ($m^3/kg$) \\ \hline
        1       &2.58       &104,8  &0.420  \\
        2       &1          &15.2   &0.827  
    \end{tabular}
\end{center}

\end{document}
